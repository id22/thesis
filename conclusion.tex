\eat{\chapter{Future work}
The goal of this thesis has been to design infrastructure services for supporting a content-dominated, highly mobile Internet. 

An overarching goal of our  future research is to design a cost-effective and high-performance infrastructure for delivering content and services that leverages resources at end users, network edge and cloud datacenters. Modern day Internet falls short of this goal in several aspects. First, the potential cost and performance benefits of cloud computing platforms are not always realized by today's applications. Second, the Internet has a limited deployment of in-network caches that are freely available to applications, and for this reason, expensive CDN services are needed to cache content near end users. Third, content delivery to mobile end hosts, an increasingly important group of users, is unsatisfactory due to various reasons including resource constraints on mobile devices and unpredictable cellular network performance. Below, We outline research topics towards addressing these shortcomings and realizing our  research goal.

\textbf{Network edge architecture:} Recent research (including our  work on NCDNs) has shown that much of the benefits due to pervasive in-network caching can be achieved by caching content only on the network edge. Still, many questions remain unanswered regarding the design of edge caches. What should be the granularity of cache placement – individual homes or a neighborhood, or a city? Which algorithms for content search are effective in locating content in nearby caches? What would be the computational and storage requirements for these caches? What, if any, changes do we need to the design of switches to interface with cache servers? Addressing these questions presents avenues for novel measurement studies, prototype development and deployment studies in organizations.

\textbf{Datacenter energy efficiency:} Energy is an important factor in the operational costs of a datacenter; therefore, energy efficiency is an important design goal for future networked systems. Recent research (including our  work on content datacenters) has shown that software-level solutions based on consolidation reduce datacenter energy use. But, many popular networked systems running inside datacenters were not designed to be energy efficient originally and hence might not interoperate well with consolidation-based energy optimization. A key open question is to evaluate the performance impact of energy optimization on various classes of networked systems such as network file systems, distributed data stores and distributed memory caches, and if needed, to re-architect these systems keeping energy efficiency as a primary design goal. We believe that this line of research would yield design principles that could be broadly applied to the design of energy-efficient networked systems.

\textbf{Mobile content delivery:} There is a pressing need to improve content delivery to mobile end hosts given the popularity of devices such as smartphones and tablets. We believe that a two-pronged approach is needed to do so. First approach is pushing content closer to end hosts by means of edge caches. The potential of edge caching inside cellular networks has not been exploited yet, and recent studies point to its benefits. Designing content delivery infrastructure inside cellular networks is an important research challenge that can potentially reduce infrastructure costs of cellular networks besides improving content delivery. Second approach is moving parts of computation from the end hosts and performing it on network edge or in cloud datacenters. But, evaluating the cost-benefit tradeoff of moving computation at an alternate location, and designing APIs that allow developers to do so are challenges that remain to be overcome.
}

\chapter{Conclusion}

This thesis focused on infrastructure service design for  a content-dominated, highly mobile Internet. We considered services for managing three types of infrastructures: an Internet service provider, a global name service and a content datacenter.

Our main contribution was to show that content placement is key to infrastructure service design and effective placement improves cost, performance, and energy-related metrics. In an ISP network, we saw that location diversity improved effective network capacity for all traffic engineering schemes, thereby blurring the difference between sub-optimal and optimal traffic engineering. In an NCDN, we saw that a simple demand-oblivious LRU caching scheme single-handedly yield network and latency costs close to the best possible even in conjunction with simple redirection and routing schemes. In Auspice, a demand-aware placement of name records resulted in significant improvents in latency and cost over existing managed DNS services, DHT-based name service designs and static placement policies. In Shrink, the effectiveness of a simple LRU caching contributed significantly to mitigating the performance impact of consolidation schemes.

Further, our work showed that, in practice, simple placement schemes perform well. Content placement problems are often NP-complete because they need to make binary decisions on whether to place content at a location or not. In practice, solving such computationally expensive problems may not be necessary because simple placement heuristics -- random placement in an ISP network, LRU caching in an NCDN and a CDC, and a simple demand-aware heuristic in Auspice -- achieve significant benefits over existing strategies and/or achieve close to optimal results as shown in this thesis.

We also studied the relative importance of optimizing placement, redirection and routing in this thesis. In an ISP network with content location diversity and a Network CDN, we showed that content placement flexibility signficantly reduces the value of a carefully engineered routing. In a global name service, a better redirection scheme does improve latencies, e.g., random-k outperforms DHT due to a better redirection scheme. However, a demand-aware placement can significantly improve cost and performance even on top of an effective redirection scheme, e.g.,  Auspice does signficantly better than random-k, which uses the same redirection scheme as Auspice, as well as managed DNS, which uses global anycast routing. Overall, these findings strengthen our thesis that placement is of key importance in infrastructure service design.
