%!TEX root = Main.tex
\begin{abstract}
Mobile devices dominate the Internet today, however the Internet rooted in its tethered origins continues to provide poor infrastructure support for mobility. Our position is that in order to address this problem,  a key challenge that must be addressed is the design of a massively scalable global name service that rapidly resolves identities to network locations under high mobility. Our primary contribution is the design, implementation, and evaluation of \auspice, a next-generation global name service that addresses this challenge. A key insight underlying \auspice\ is a {\em demand-aware} replica {placement engine} that intelligently replicates name records to provide low lookup latency,  low update cost, and high availability.  We have implemented a prototype of \auspice\ and compared it against several commercial managed DNS providers as well as state-of-the-art research alternatives, and shown that \auspice\ significantly outperforms both. We  demonstrate proof-of-concept that \auspice\ can serve as a complete end-to-end mobility solution as well as enable novel context-based communication primitives that generalize name- or address-based communication in today's Internet.

%We have implemented a prototype of \auspice\ and evaluated it on Planetlab, a local cluster, as well as through large-scale trace-driven experiments. Our experiments show that \auspice\ provides 1.0 sec to 24.7 sec lower update latencies than commercial managed DNS services and up to 9$\times$ lower lookup latencies than a popularity-based DHT replication scheme. 

%well as  to DNS by up to an order of magnitude.

%which places replicas of name resolvers close to pockets of demand so as to optimize user-perceived response times while respecting capacity and consistency constraints. Furthermore, we also develop simple heuristic algorithms that achieve response times close to the optimal. To demonstrate the generality of \auspice's service replica placement engine, we conduct several case studies including a name resolution workload based on traces from Alexa and a large large commercial CDN; a Twitter stream service; and a compute-intensive database querying service. Our experiments based on an implemented prototype of \auspice\ over Planetlab and Amazon EC2 as well as larger-scale trace-driven experiments show that \auspice\ achieves significant improvements in user-perceived response times, in some cases exceeding an order of magnitude compared to state-of-the-art solutions.

\end{abstract}
