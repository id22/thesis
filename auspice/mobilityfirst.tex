%!TEX root = New.tex
\section{Naming in MobilityFirst}
\label{sec:MF}

In this section, we describe the design of the naming subsystem in MobilityFirst, which forms an important part of the motivation for \auspice. %Readers uninterested in future-Internet-architectural usage scenarios can skip to Section \ref{sec:auspice} with minimal loss of continuity.

A {\em name} in MobilityFirst is a {\em globally unique identifier} (GUID) that can be used to identify a variety of {\em principals} such as an interface, a device (or the set of all interfaces on the device), a service (e.g., HTTP), an end-user (or the set of all devices owned by the user), a network location, or named content. 

\textbf{Certification, addressing, and routing.} A GUID is self-certifying, i.e., any principal can authenticate another principal claiming a GUID without the need for third-party certification. A self-certifying GUID is derived simply as a one-way hash of a public key, so a GUID can be authenticated using just a bilateral challenge/response step as only the authentic principal possesses the corresponding private key. 

Self-certifying identifiers do not obviate certification authorities. In addition to a GUID, a principal may be assigned an optional human-readable name (e.g., ``www.amazon.com'' or ``Tom Sawyer's cell phone'')  or an inexact intent (e.g., a set of search keywords or other abstract descriptions). In such cases, {\em name certification services} bind the human-readable name or intent to a public key, and end-users or applications must first obtain such a certificate from a service they trust.

A {\em network address} (NA) is a self-certifying identifier for a network or an autonomous domain. The tuple [GUID, NA] is a routable destination address carried in packet headers. Senders query the name service to obtain the NA corresponding to a GUID (much like they query DNS to obtain the current IP address corresponding to a domain name) before sending the first packet to the destination. Senders are also permitted to send a packet addressed just to a GUID, thereby implicitly delegating to the first-hop router the task of querying the name service for the corresponding NA. End-to-end packet forwarding is accomplished by an interdomain routing protocol that delivers packets to the destination NA (oblivious of the destination GUID) and subsequently by an intradomain routing protocol that delivers the packet to the destination GUID.

\textbf{Hierarchical routing.} The interdomain routing protocol enables reachability to NAs much like the current Internet enables reachability to IP prefixes. Thus, the number of forwarding table entries in a core router is the total number of NAs. As the number of NAs may grow significantly over time (e.g., home networks, vehicular networks, body area networks, etc.), the interdomain routing protocol is designed to support a small number of levels of hierarchy so as to trade off packet header space against forwarding table size. An interdomain routing hierarchy with two levels as in our current implementation consists of  {\em core} and {\em edge} networks. The name service translates a GUID into a two-tuple $[X, T]$ where $X$ is the last core network enroute to $T$, the terminal network to which the GUID is attached. Core networks only maintain forwarding entries for other core networks and a small number of their ``customer'' edge networks while edge networks need maintain forwarding entries only for a small number of their ``provider'' core networks and edge networks in their vicinity.

\textbf{Multihoming.} A multi-homed GUID is simultaneously attached to more than one core or terminal network. In these cases, the name service by default returns a list of all homes of the form: $\{[X_1, T_1], [X_2, T_2], \ldots \}$. Multi-homed GUIDs can also specify expressive policies for engineering incoming traffic, e.g., prefer WiFi to 3G; or WiFi for delay-tolerant downloads and 3G for delay-sensitive traffic, etc. Network operators can explicitly create NAs for portions of their network and specify incoming traffic engineering policies instead of abusing longest-prefix matching as commonly practiced today.

\textbf{Content retrieval.} Content in MobilityFirst is also named using a self-certifying identifier, but a content GUID is simply the hash of the content itself. This widely used technique \cite{BitTorrent} obviates the need for public keys for verifying the authenticity of static content. Given a content GUID, the name service returns a list of network addresses $\{[X_1, T_1], [X_2, T_2], \ldots \}$ from where replicas of a content may be fetched. The list of these locations typically only includes replicas maintained by content providers or their delegates, however any network intermediary can intercept a content request and serve the content if it possesses a cached copy.

\textbf{Indirection and grouping.} Indirection enables the name service to resolve a GUID to another GUID and grouping allow a set of GUIDs to be a assigned a single GUID. Indirection and grouping are powerful operations that enable GUIDs to support new network primitives in an efficient manner. We illustrate these benefits using three examples: 

(1) {\em Multicast}: A multicast GUID (MGUID) has the same format as a regular GUID and the resolved output of the name service has the same format as multi-homed network address. However, the resolution procedure and routing are different as follows. The name service maintains a membership set for each MGUID that consists of all GUIDs subscribed to the multicast group. Each member GUID $i$ in MGUID subscribes to the group via a single home, $NA_i$. The name service resolves an MGUID by returning the union of all $NA_i$'s that have at least one GUID subscribed to the MGUID. By default, the sender is responsible for sending data addressed to $[\textit{MGUID}, NA_i]$ for each of the returned $\textit{NA}_i$'s. When packets arrive at a destination $\textit{NA}_i$, the $\textit{NA}_i$ is responsible for resolving the MGUID to the subset of member GUIDs attached to its network and forwarding a copy to each member. Geo-casting, e.g., sending a message to all yellow cabs near Times Square, and other context-aware services can be supported in a similar manner by having the name service maintain geolocation or context attributes in addition to network locations of GUIDs.

(2) {\em Content directories}: Content is typically organized in hierarchical name spaces, e.g., www.nytimes.com/sports, www.nytimes.com/business, and so on, to enable grouping and colocation of related content. However, as GUIDs do not capture locality, moving the location of a large content directory from one network domain to another will by default result in updates to all of the constituent content GUIDs. To reduce this overhead, a set of content GUIDs can be assigned a directory GUID, in which case the name service maintains network addresses only for the directory GUID and returns it upon a request for any constituent content GUID.

(3) {\em Grouping}: Affinity groups as for content above help reduce the overhead of maintaining network locations for any group of GUIDs, e.g., a group of interfaces in an airplane, even when no data transfers are actually taking place. Thus, any group of colocated GUIDs can be assigned a group GUID that requires only one update for the entire group every time the network location of the group changes.

\textbf{Access control:} The name service controls access to network addresses as well as other attributes (such as geolocation, traffic engineering preferences, etc.) of a GUID by allowing the owner (or any entity possessing the corresponding private key) to specify access control policies for the GUID's stored attributes. The owner can specify either a blacklist or whitelist for read or write access to each stored attribute.
