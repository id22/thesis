%!TEX root = New.tex
\section{Deployment}

This section discusses deployment possibilities of \auspice\ in a multi-provider scenario, unlike the single provider scenario described earlier in the paper. Existence of multiple providers of \auspice\ would foster competition resulting in better quality of service, and would give a customer the freedom to choose the provider she considers most trustworthy. Next, we  discuss two approaches for a federated multi-provider deployment of \auspice.

%Until now, we have discussed how a single deployment of  \auspice\ is managed. However, there are advantages of having multiple providers of a global name resolution service. First, existence of multiple providers fosters competition and results in better quality of service overall. Second, it gives freedom to a customer, e.g., a website or a smartphone owner, to choose the provider she considers most trustworthy. Below, we discuss three approaches of deploying \auspice\ in a federated manner with multiple providers along with their pros and cons.

(1) Use two-level hierarchical names. The first part of a name identifies the \auspice\ provider, and the second part identifies the entity being named. Clients maintain lists of name servers for each \auspice\ provider to support name resolution.  When a client needs to resolve a hitherto unknown provider name, it requests its parent provider (provider who services the client's name), which returns the list of name servers for that provider. This federation scheme is very simple and only requires that all providers maintain lists of name servers of other providers. The drawbacks of hierarchical names are that they must be changed whenever a customer switches providers and they preclude the use of self-certifying identifiers \cite{ROFL} as names.

%(2) A client requests all providers (one-by-one or in parallel) to find out which provider services a name. This approach lets names be arbitrary strings, but increases the load on an \auspice\ provider compared to an un-federated deployment. A provider can receive a request for every name that it does \emph{not} potentially all clients. 

(2) A provider maintains lists of (flat) names serviced by other providers. These lists are populated over time based on requests received by a provider.  A client first queries a name server in its parent provider that is expected to be a replica-controller for that name. For a name that is \emph{not} served by the parent provider, the name server returns the address of a name server of the provider servicing the name; the client is subsequently redirected to an active replica by that provider. To determine which provider services this name, a name server first checks its locally stored lists. Otherwise, it obtains the name of the provider by querying other providers and updates its lists based on responses received. 
This federation approach requires more effort on part of the provider, but enables names to be defined as arbitrary strings, such as a self-certifying identifier.

%The number of requests received by a provider for names that it does \emph{not} service is proportional to the number of its customers, which we believe is a fair division of responsibility among providers. Further, this approach allows names to be chosen as arbitrary strings.

%While this federation scheme is not as simple as (1), its allows names to be chosen as arbitrary strings.

%For the flexibility in choosing names that this approach provides, it adds a small amount of federation overhead among providers.



% and adds a smaller overhead of federation on each provider compared to (2). The number of requests received by a provider for a name its does \emph{not} service is proportional to the number of providers, which is expected to be a small number. Therefore, we believe this approach provides the best trade-off between overhead of federation and naming flexibility among the three approaches.


%A client needs to determine the provider only the first time the client resolves a name. The provider for the name can be cached by the client for later requests. 
