%!TEX root = New.tex
\begin{abstract}
A placement strategy determines the locations at which content is placed in a network. An effective placement strategy improves availability of content in presence of failures, and makes judicious use of available resources, such as storage and bandwidth, to reduce the latency of content accesses.

While effective placement is known to improve content delivery in the Internet, this thesis studies the impact of placement strategies on the objectives of the underlying network. Our work is closely related to prior research on the interaction between network and content delivery. To our knowledge, this is the first effort to study how placement strategies shape this interaction and affect the objectives of networks and content delivery systems. We study this interaction in three scenarios: an Internet service provider (ISP) network in which all content is placed at a small number of randomly chosen locations, an ISP that controls both the underlying network and content delivery on its network and therefore has full flexibility in designing placement strategies, and a datacenter of a content delivery network (CDN) whose goal is energy-minimization, and makes both placement and routing decisions to that end. Based on experiments with real ISP and datacenter topologies and content access traces from a leading CDN, we show that content placement is a powerful factor in shaping the network traffic and that simple placement strategies are effective in improving cost-, performance- and energy-related metrics for networks and content delivery systems.

Several applications today generate dynamic content such as stock prices, weather information, and status updates posted on social media websites. Placement of dynamic content is a challenging problem due to a fundamental cost-vs.-performance trade-off: dynamic content replication at multiple locations improves latency of content accesses but increases update propagation costs. We design and implement Auspice, a system for placement of dynamic content across geo-distributed data centers. Auspice infers pockets of high demand for a name and uses a heuristic placement strategy to provide low request latency, low update cost, and high availability. An application suited for Auspice is a global name service to store name-to-address mappings of mobile devices. We extensively evaluate Auspice for an expected workload of such a global name service and show that it significantly outperforms both commercial managed DNS services as well as DHT-based replication alternatives to DNS. 


%In the latter part of the thesis, we focus on enhancing network support for content delivery to mobile end-hosts. Content delivery requires initiating and maintaining connections to mobile hosts, and in turn requires a global name service that translates host names of mobile devices to their network addresses under frequent mobility. Our contribution is the design and implementation of a scalable, geo-distributed name service Auspice that meets this challenge. The key insight underlying Auspice is a placement engine that replicates name records globally to provide low lookup latency, low update cost, and high availability.  Our evaluation shows that Auspice significantly outperforms both commercial managed DNS services as well as DHT-based replication alternatives to DNS.

%
%end-host mobility.  Today, content delivery systems need to work around the problem of being unable to initiate and maintain connections to mobile hosts by building application-level mechanisms such as push notification services.
%Our position is that end-host mobility can be seamlessly handled with the support of a global name service that translates host names to their network address under frequent  mobility of end-hosts.
%A key decision for this service, affecting both performance and update costs, is the placement of name records of mobile hosts. Our name service design, Auspice, infers pockets of high demand for a name and uses a heuristic placement scheme to provide low address lookup latency, low update cost, and high availability. Our evaluation of Auspice's implementation  shows that Auspice significantly outperforms both commercial managed DNS services as well as DHT-based replication alternatives to DNS.


%First, we consider a scheme that places content at a small number of randomly chosen locations in an Internet service provider (ISP) network; end-users leverage the location diversity of content by downloading in parallel from all locations. We compare several route optimization schemes for an ISP accounting for the location diversity of content. Our experiments on real topologies and traffic matrices show that the ability to download content from just 2-4 locations enables all schemes to achieve near-optimal capacity to tolerate surges in traffic demand, and even simple static routing to be within 30\% of optimal.
%
%Second, we model a single entity that controls both network and content delivery, as in the case of a network CDN (NCDN). An NCDN can independently or jointly optimize both placement and routing. Further, it can make these decisions in ``planned" manner based on recent history of workloads. Our experiments on real network topologies and traces from a large CDN surprisingly show that  ``unplanned'' placement schemes, e.g., LRU caching, combined with a simple static routing outperform (sometimes significantly)  realistic (i.e., history-based) joint-optimization schemes and can achieve network cost and user-perceived latencies close to those of a joint-optimal scheme with future knowledge.
%
%Third, we propose to study energy-minimizing schemes for networks seeking to establish that content placement flexibility helps reduce energy consumption for NCDN and datacenter networks. Our central idea is that requests for content can be served by placing content on a subset of nodes during off-peak hours and if these nodes are carefully chosen, it enables large fraction of switches and links to be turned off with a minimal degradation in user-perceived performance. We present preliminary results from trace-driven experiments to support our hypothesis.
%
%Finally, we focus on enhancing network support for end-host mobility.  Today, content delivery systems need to work around the problem of being unable to initiate and maintain connections to mobile hosts by building application-level mechanisms such as push notification services.
%Our position is that end-host mobility can be seamlessly handled with the support of a global name service that translates host names to their network address under frequent  mobility of end-hosts.
%A key decision for this service, affecting both performance and update costs, is the placement of name records of mobile hosts.
%Our name service design, Auspice, infers pockets of high demand for a name and uses a heuristic placement scheme to provide low address lookup latency, low update cost, and high availability.
%Our evaluation of Auspice's implementation shows that Auspice significantly outperforms both commercial managed DNS services as well as DHT-based replication alternatives to DNS.

%The locations at which a global name service places the name records of a mobile hosts is a key decision for this service, affecting both performance and update costs of the service. 


%on multiple testbeds and via trace-driven experiments

%problems faced by content delivery systems due to lack of network support for host mobility.
% to provide functionality such as  mobile-to-mobile data transfer, and server-to-mobile push notifications.
%It is not possible today to initiate and maintain connections to mobile hosts because there is no network support for obtaining the network address of a mobile host.  
%Hence, content delivery systems build application-level mechanisms to provide functionality such as  mobile-to-mobile data transfer, and server-to-mobile push notifications.


%Our experiments on real network topologies and traces from a large CDN surprisingly show that (1) realistic (i.e., history-based) joint-optimization schemes offer little benefit (and often significantly underperform) compared to simple, �unplanned� schemes for routing and placement and (2) simple, unplanned schemes suffice to achieve network cost and user-perceived latencies close to those of a joint-optimal scheme with future knowledge. 

%(1) realistic (i.e., history-based) joint-optimization schemes offer little benefit (and often significantly underperform) compared to simple, �unplanned� schemes for routing and placement and (2) simple, unplanned schemes suffice to achieve network cost and user-perceived latencies close to those of a joint-optimal scheme with future knowledge.


% can make it feasible initiate and maintain connections to mobile hosts.
%Today, content delivery s
%
%As a consequence, 
%
%Our position is that a global name service that maps host names to their network address under frequent host mobility can make it feasible initiate and maintain connections to mobile hosts.  The key challenge in designing such a service is placement of name records of mobile hosts provide low address lookup latency, low update cost, and high availability. 
%
%
%The key challenge in designing a service that maps host names to their network address under frequent host mobility
%
%The key challenge in designing such an infrastructure is placement of 
%
%We have designed and implemented a global name service, Auspice, that maps host names to their network address under frequent host mobility.
%
%
%Our contribution is the design and implementation of a global name service, Auspice, that maps host names to their network address under frequent host mobility.
%The key insight underlying Auspice is a placement engine that replicates name records of mobile hosts to provide low address lookup latency, low update cost, and high availability. Our evaluation of Auspice on multiple testbeds and via large-scale trace-driven experiments shows that Auspice significantly outperforms both commercial managed DNS services as well as DHT-based replication alternatives to DNS.
%
%The key challenge we address is that of placement of name records  of mobile hosts on geo-distributed servers that achieves low address lookup latency, low update cost, and high availability.
%
%% underlying Auspice is a placement engine that replicates name records of mobile hosts to provide low address lookup latency, low update cost, and high availability.
%
%
%%Finally, we focus on improving support for  end-host mobility so that 
%
%%Finally, we focus on enhancing network support for end-host mobility. If it becomes feasible to initiate and maintain connections to mobile hosts, content delivery systems can use rich communication patterns, e.g., mobile-to-mobile data transfer, and server-to-mobile push notifications, without building application-level mechanisms to do so. 
%
%
%
%%, the lack of which is restricting content delivery systems to follow a single communication pattern where connection initiation happens only from the end-host to the content delivery systems.
%
%
% request-response 
% so that content delivery systems 
% can use richer communication patterns, e.g., mobile-to-mobile data transfer, and server-to-mobile push notifications, without building application-level mechanisms to do so. To make it feasible to initiate and maintain connections to mobile hosts, 
%













%Second, we consider a content delivery service operated by an ISP on its network, called a network CDN�(NCDN). As an NCDN has full control over content placement and routing, it can independently or jointly optimize both these decisions. We perform experimental analysis, based on traces from a large content distribution network and real ISP topologies, and show that (1) realistic (i.e., history-based) joint optimization strategies offer little benefit (and often significantly underperform) compared to simple, �unplanned� strategies for routing and placement and (2) simple, unplanned strategies suffice to achieve network cost and user-perceived latencies close to those of a joint-optimal strategy with future knowledge.


%The growth of Internet traffic fueled by online video, new mobile devices, and ever-expanding web services presents a challenge to content delivery networks (CDNs) delivering content via their server infrastructure and internet service providers (ISPs) carrying the traffic on their core network. 
%Towards the goal of providing high-performance content delivery in a cost-effective manner, this thesis studies resource management strategies both at the CDN-level and at the ISP-level. 
%
%We study two approaches by which ISPs can increase effective network capacity: 
%(1) Route optimization: While route optimization is a well-known technique, we present a re-evaluation of several techniques based on user-perceived metrics   taking into account the location diversity in the Internet.
%(2) Integrating CDN functionality: An ISP offering CDN services (called network CDN) can reduce network costs via content caching and generate additional revenue.  We evaluate several schemes for managing content delivery and route optimization in network CDNs. Rather surprisingly, we find that simple techniques for routing and content delivery perform well for a variety of workloads and metrics.
%
%In the context of CDNs, we study two approaches that respectively enable distributing content (especially larger-sized files) with minimal server bandwidth, and  dynamic content to be served with small latency and a small update cost.
%(1) Hybrid CDN: A hybrid-CDN design reduces bandwidth costs to distribute content by leveraging clients� upload bandwidth, which supplements the server bandwidth contributed by CDN. We study how a hybrid CDN should manage its sever bandwidth across content being delivered to optimize performance/cost objectives. 
%(2) Dynamic content: We design locality and load-aware replication strategies for dynamic content, which reduce the cost of keeping dynamic content updated by restricting the amount of replication, and yet achieves small response latencies by replicating content in a locality-aware manner.
%
%
%In this process, we consider two approaches to resource management: (1) planned approach, which assumes that content workloads in future will resemble those in the past, and allocates resources globally using an offline analysis of past workloads, and (2) unplanned approach, which manages resources either using a fixed policy or using an online algorithm based on local information. 
%Potentially, a planned approach can outperform an unplanned approach due to a global optimization of resources and a long-term knowledge of workloads. However, we find that simple unplanned approaches perform well in many cases. In some cases, a planned approach indeed gives better performance than an unplanned approach, but a planned approach is not always effective due to unpredictability of the workload or because it is infeasible to estimate the optimal solution or even its approximation.
%
%
%
%
%\eat{
%
%We study two approaches by which ISPs can increase effective capacity of their network in a cost-effective manner. First  is to use route optimization techniques and second is to integrate content delivery functionality into their networks. Route optimization is a well-known technique, but existing work has largely ignored how route optimization affects end-user performance and is affected by the location diversity in the Internet. We present a re-evaluation of several route optimization techniques in the light of these observations. Integrating content delivery functionality into ISP networks reduces  network costs due to content caching and the CDN service generates additional revenue for ISPs. Such ISPs, called Network CDNs, mange content delivery in addition to optimizing routes, and face several choices, such as whether to jointly or independently optimize routing and content delivery, and which technique to use for each of them.  Using traces from a large CDN and real ISP topologies, we present a comprehensive evaluation of several schemes for  managing network CDNs. Rather surprisingly, we find that simple techniques for routing and content delivery perform well for a variety of workloads and metrics.  
%
%
%
%
%
%Content delivery networks exhibit great diversity in their architectures, in terms of the technology, the geographical spread, and the infrastructure. For example,  a leading CDN Akamai operates a distributed cache across thousands of locations while other CDNs such as Limelight and Level-3 operate from tens of locations only.  The diversity in the architectures of content delivery networks poses challenges and provides opportunities that are  unique to each type of CDN. Two CDN architectures that that best exemplify this diversity are  Network CDNs (NCDNs) and Hybrid CDNs (HCDNs). % how did they come about? evolution!
%
%The popularity of CDNs, P2P applications, and mirrored cloud hosting in the Internet gives rise to a phenomenon that we term \emph{location diversity}, or the ability to download content from multiple locations. This thesis begins with a empirical analysis, which shows that the effect of application-level adaptation enabled by \emph{location diversity}  affects the network-level traffic engineering done by ISPs, reducing its importance to the extent that all existing traffic engineering schemes achieve nearly the same network capacity as the ideal scheme with perfect knowledge of traffic demand. 
%
%
%%We begin with a empirical analysis of the effect of \emph{location diversity} in the Internet. Location diversity, or the ability to download content from multiple locations,  is made possible by the presence of CDNs, P2P applications, and mirrored cloud hosting.  Our analysis shows that application-layer adaptation to location diversity influences traffic patterns, thereby affecting network-level traffic engineering done by the ISPs to the extent that all existing traffic engineering schemes achieve nearly the same network capacity as the ideal scheme with perfect knowledge of content demand. 
%
%An NCDN is a CDN that owns the underlying network as well as the content delivery infras	tructure. NCDNs make traffic engineering decisions like a traditional ISP and content placement and routing redirection decisions like a traditional CDN. Network cost optimization is a key concern for an NCDN because it keeps network congestion low and reduces infrastructure cost in the long run.  However, the unique feature of an NCDN, its control over content distribution, enables it to create location diversity on demand and ``shape'' traffic to its advantage and thereby lower its network cost significantly.  Experimental analysis, based on Akamai CDN traces and real ISP topologies, presented in this thesis  suggests that location diversity in an NCDN context is so much powerful that NCDNs can achieve near-optimal network cost with a simple static routing and thereby sideline all traditional traffic engineering schemes. 
%
%
%A HCDN opportunistically leverages client resources, especially their uplink bandwidth, to augment their managed infrastructure resources, to provide performance in a scalable and cost-effective manner. A key challenge for HCDNs 
%
%In this thesis, we focus on a key component of CDN costs, 
%
%A HCDN architecture is attractive due to its lower infrastructure and server bandwidth costs but realizing bandwidth cost savings and 
%
%
%A key challenge in client-assisted content delivery is determining how to allocate limited server bandwidth across a
%large number of �les being concurrently served so as to optimize global performance and cost objectives.
%
%}
%% However, an NCDN can optimize network cost by effective content placement and request redirection 
%
%%Our empirical analysis of NCDNs shows that NCDNs can achieve near-optimal network costs with a simple static routing and thereby sidelining the importance of traditional traffic engineering. 
%%
%%As location diversity 
%%
%%
%%This is because  an NCDN has complete control over content placement and request redirection which enables it to ``shape'' traffic to its advantage and  thereby achieve significantly lower network cost. 
%%
%%
%%However, an NCDN has complete control over content placement and request redirection which enables it to ``shape'' traffic to its advantage and  thereby achieve significantly lower network cost. 
%%
%%Our empirical analysis of NCDNs shows that NCDNs can achieve near-optimal network costs with a simple static routing and thereby sidelining the importance of traditional traffic engineering.  This is because an NCDN is in a powerful position to 
%%
%%intelligent content placement and request redirection
%%
%%intelligent content placement and request redirection achieve significant cost reduction for NCDNs and thereby marginalize the role of traditional traffc engineering.
%%
%%
%%
%%
%%
%%
%%In this thesis, we present an empirical evaluation of content placement and traffic engineering strategies for an NCDN. 
%%
%%An NCDN which has complete control over content placement can create 
%%
%%
%%Our results show that an NCDN using simple content placement and simple static routing strategy suffice to achieve near-optimal network cost. 
%%
%%A central thesis of this paper is that intelligent content placement and request redirection achieve signicant cost reduction for NCDNs and thereby marginalize
%%the role of traditional trac engineering.
%%
%%Our results show that 
%
%
%
%
%
%
%
%
%
%
%


\end{abstract}
