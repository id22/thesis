%!TEX root = New.tex
\begin{abstract}
Most of the Internet traffic is content, and most of the Internet connected hosts are mobile. Our work focuses on the design of infrastructure services needed to support  such a content-dominated, highly mobile Internet. In the design of these services, three sets of decisions arise frequently: (1) \emph{content placment} for selecting the locations where a content is placed, (2) \emph{request redirection} for selecting the location where a particular request is served from and (3) \emph{network routing} for selecting the physical path between clients and the services they are accessing. Our central thesis is that content placement is a powerful factor, and is often more powerful than redirection and routing, in determining the cost, performance and energy-related metrics for these services. In support of this thesis, we consider three types of infrastructure.

\textbf{Internet service provider (ISP):} In an ISP carrying content-dominated traffic, we show that combinations of simple placement and routing schemes are effective in optimizing an ISP's performance and cost objectives. Further, we show that effective content placement contributes more than optimizing network routing to achieve an ISP's objectives. Our findings question the value of traditional ISP traffic engineering schemes that optimize routing alone, while simplifying the task of traffic engineering for the operators.

\textbf{Global name service (GNS):} We design and implement a GNS, \auspice, that resolves names to network addresses for highly mobile entities, thereby providing a key building block for establishing communication between mobile entities in the Internet. A key distinction between \auspice\ and other name services is a {\em demand-aware} replica {placement engine} that intelligently replicates name records to provide low lookup latency,  low update cost, and high availability. In our experiments, Auspice's placement scheme enables it to significantly outperform commercial managed DNS providers, DHT-based replication as well as static placement schemes that use the same redirection scheme as Auspice.

\textbf{Content datacenter (CDC):} Content datacenters cache and serve content to improve user-perceived performance for content accesses. In a CDC, we quantify the tradeoff between energy savings via consolidation and the user-perceived performance impact based on a real CDC workload. A key reason for the small user-perceived performance impact we observe is that despite server consolidation, a simple placement scheme achieves a cache hit rate close to that of an unconsolidated datacenter. We propose a new network-aware server consolidation approaches that enables additional network energy savings over a network-unaware server consolidation for common datacenter topologies.

\end{abstract}
