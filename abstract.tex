%!TEX root = New.tex
\begin{abstract}
Most of Internet traffic is content, and most of Internet connected hosts are mobile. Our work focuses on the design of services needed to support  such a content-dominated, highly mobile Internet. In the design of these services, three sets of decisions arise frequently: (1) content placment, selecting the locations where a content is placed, (2) request redirection, selecting the location where a particular request is served from and (3) network routing, selecting the physical path between clients and the services they are accessing. Our central thesis is that content placement is a more powerful factor that request redirection and network routing in determining the cost, performance and energy-related metrics for these services. To that end, we consider three types of infrastructure. 

\textbf{Internet service provider:} In an Internet service provider carrying content-dominated traffic, we show that combinations of simple placement and routing schemes are effective in optimizing operators? performance and cost objectives. Further, I show that effective content placement contributes more than optimizing network routing to achieve an ISP's objectives. Our findings question the value of traditional ISP traffic engineering schemes that optimize routing alone, while simplifying the task of traffic engineering for the operators.


\textbf{Global name service:} We design and implement a global name service that resolves names to network addresses for highly mobile entities, thereby providing a key building block for establishing communication between mobile entities in the Internet. A key insight underlying \auspice\ is a {\em demand-aware} replica {placement engine} that intelligently replicates name records to provide low lookup latency,  low update cost, and high availability. In our experiments, Auspice's placement strategy enables it to outperform commercial managed DNS providers as well as state-of-the-art research alternatives.


Content datacenters (CDCs), used for caching and serving content to end-users, play a significant role in improving user-perceived performance for content accesses. In a CDC, we quantify the tradeoff between energy savings via consolidation and the user-perceived performance impact based on a real CDN workload. A key reason for the small user-perceived performance impact is that despite server consolidation, a simple placement strategy achieves a cache hit rate near to an unconsolidated datacenter.

\end{abstract}
