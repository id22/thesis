


\section{Model}
%\label{sec:optimize}


%\textcolor{blue}{
%Content-aware traffic engineering for an NCDN refers to the combined task of content distribution and traffic engineering. There are several content-aware traffic engineering strategies that an NCDN can deploy. An NCDN can use any combination of demand-aware or demand-oblivious content distribution and routing strategies ($\S$\ref{sec:background})  as its content-aware traffic engineering strategy. An NCDN can also jointly optimize both content distribution and traffic engineering in a demand-aware manner.
%}



To answer the above questions, we develop an optimization model for NCDNs to decide placement, routing, and redirection so as to optimize network cost or user-perceived latency. We formulate the optimization problem as a mixed-integer program (MIP), present hardness and inapproximability results, and discuss approximation heuristics to solve MIPs for realistic problem sizes.


%\textcolor{blue}{
%This section presents a optimization strategy for NCDNs to jointly optimize content placement, routing, and redirection. First we describe our NCDN model and cost functions followed by a description of the optimization formulation as a mixed-integer program (MIP). Next, we present hardness and inapproximability results for this problem. Finally, we discuss approximation techniques that we use to solve our MIP formulation. 
%}



\eat{
In this section, we formalize the \ncp\ model and the problem of determining the optimal content placement and routing as a mixed integer program (MIP).  
Next, we present hardness and inapproximability results for this problem. Finally, we discuss approximation techniques that we use to solve our MIP formulation. 
}

%This formulation takes as input a content matrix, i.e., the demand for each content at each network point-of-presence (PoP), and  computes a placement and a routing that minimizes the maximum link utilization (MLU) while respecting link capacity and storage constraints.

%Further, we present a variant of this MIP that determines the optimal placement strategy for a given routing scheme. 
%We also present a variant of the MIP that determines the optimal placement strategy for a given routing scheme



%. We also present two  variants of this formulation: (1) a MIP that determines the optimal placement strategy for a given routing scheme; and (2) a linear program that optimizes routing for a given placement strategy. All of these formulations take as input a content matrix, i.e., the demand for each piece of content at each network point-of-presence (PoP), and seek to compute a placement and/or a routing strategy that minimizes the maximum link utilization (MLU) while respecting link capacity and storage constraints. 

%In this section, we formalize the \ncp\ model and present three strategies to enable placement-routing-redirection in \ancp.  First, we present a mixed integer program (MIP) which calculates  content placement and routing to optimize network cost based a content matrix, i.e., the demand for each piece of content at each network point-of-presence (PoP). 
%Next, we explain why content chunking, i.e., delivering content in small chunks, can improve performance of  both demand aware and demand oblivious content placement. Finally, we describe a link load aware request redirection strategy for a demand oblivious placement, designed to reduce \ancp's network cost.




%Next, we describe a link load aware request redirection strategy for a demand oblivious content placement can use to reduce \ncp\ network cost. 


%In this section, we formalize the \ncp\ model and the problem of determining the optimal content placement and routing strategy as a mixed integer program (MIP). We also present two  variants of this formulation: (1) a MIP that determines the optimal placement strategy for a given routing scheme; and (2) a linear program that optimizes routing for a given placement strategy. All of these formulations take as input a content matrix, i.e., the demand for each piece of content at each network point-of-presence (PoP), and seek to compute a placement and/or a routing strategy that minimizes the maximum link utilization (MLU) while respecting link capacity and storage constraints. Table \ref{table:paramtable} lists all the input parameters and  variables used in the optimization formulations in this section.

%In this section, we describe our solutions to the traffic engineering strategies discussed in Section \ref{sec:tescheme}. Our primary contribution is a MIP formulation which jointly optimizes the content placement and routing in the network based on network parameters, content size and content demand at each node. First we present our Network-CDN model, followed by a MIP formulation for optimal traffic engineering in our model. Finally, we discuss our solutions for TE schemes other than the optimal TE scheme. 


\begin{table*}[t]
\centering
\begin{small}
%\begin{center}
\begin{tabular}{|p{6.15in}|} 
 \hline
\textbf{Input variables and descriptions}\\ 
\end{tabular}
\begin{tabular}{|p{1in}|p{5in}|}
\hline
  $V$ & Set of nodes where each node represents a PoP   \\
  $E$ & Set of edges where each link represents a communication link   \\
   $o$ & Virtual origin node that hosts all the content in $K$  \\ 
%  $ X(o)$ & $X(o) \in V$  s.t., $o \in O$ is connected to $X(o)$ with infinite capacity link.\\ 
   % $ X$ & $X  \subset V$.   Each $x\in X$ is connected to unique $o \in O$  and vice versa.\\ 
  $ X$ & Set of exit nodes in $V$ \\ 
  $A_i$ & Number of servers at node  $i \in V$ \\
  $D$ & Storage capacity per server  (in bytes) \\
  $N$ & Maximum  throughput of one server (in bits/sec) \\
  $C_e$ & Capacity of link $e \in E$ (in bits/sec)\\
  $K$ & the set of all content accessed by end-users \\
  $S_k$ & Size of content $k \in K$.   \\
  $T_{ik}$ & Demand (in bits/sec)  at  node $i \in V$ for content $k \in K$ \\
  $P$ & Energy use of one server (in watts)\\
$R$ & Energy use of router chassis at each node (in watts)\\
$L_e$ & Energy use of line card at router connecting link $e \in E$ (in watts)\\
 $\alpha$ & Upper bound on the MLU of the network \\
  % $P(i)$ & Set of  outgoing links at PoP $i \in V$\\
   %$Q(i)$ & Set of  incoming links at PoP $i \in V$\\
   \hline
   \end{tabular}
\begin{tabular}{|p{6.15in}|}
  \hline
\textbf{Decision variables and descriptions}\\ 
\end{tabular}
\begin{tabular}{|p{1in}|p{5in}|}
\hline
$v_i$ & Number of servers turned on at  node $i \in V$\\
$y_e$ & Binary variable indicating whether link $e \in E$ is active \\
$w_i$ & Binary variable indicating whether router chassis at node $i \in V$ is turned on \\
$z_k$ & Binary variable indicating whether one or more copies of content $k$ is placed in the network\\
$x_{jk}$ & Binary variable indicating whether content $k$  is placed at node $j \in V \cup \{o\}$\\

$f_{ij}$ & Total traffic from node $j$ to node $i$\\
$f_{ije}$ & Traffic from node $j$ to node $i$ crossing link $e$.\\
$t_{ijk}$ & Traffic demand at node $i \in V$ for content $k\in K$ served from node $j \in V \cup \{o\}$ \\\hline
\end{tabular}
\end{small}
\label{table:paramtable}
\caption{List of input  and decision variables for the \ncp\ problem formulation.}
\vspace{-0.2in}
\end{table*}

\subsection{\ncp\ Model}
\label{sec:model}


%\Ancp\ consists of a set of nodes $V$ where each node represents a PoP in the network. 

Table 1 lists all the model parameters. \Ancp\ consists of a set of nodes $V$ where each node represents a PoP in the network. The nodes are connected by a set of directed edges $E$ that represent the backbone links in the network. The set of content requested by end-users is represented by the set $K$ and the sizes of content are denoted by $S_k, k\in K$.  The network resource constraints is the link capacities $C_e, e\in E$.  \textcolor{blue}{Each node is has $A_i$ servers deployed, each of which has a storage capacity $D$ and a maximum throughput $N$. }

%{\color{blue}We model the server resources at each PoP in terms of the disk capacity $D_i, i \in V$, (mnemonic D = disk)  and the maximum network throughput, $N_i$,  (mnemonic N = network) which the servers at PoP $i$ can deliver.} 



\textcolor{blue}{We adopt commonly used power consumption model of backbone routers. The energy draw of a backbone router is the sum of energy consumption of the chassis  and of the line cards plugged into the chassis.  All routers have a similar chassis which consumes $R$ units of energy, the line card for link $e \in E$ uses $L_e$ energy units on routers at both ends.}

\textcolor{blue}{Energy consumption of a server in use in $P$ watts. At any time, only a subset of servers at a PoP are turned on at any time to reduce energy use. Total energy use of servers at a PoP is $P$ times the number of servers that are turned on.}

%\textcolor{blue}{The server resources at  PoPs are assumed to be energy proportional.  The maximum total energy use of servers at a PoP is $P_i, i \in V$, and their energy use is proportional to the utilization of servers at the PoP.}

%The energy consumption of these components depends on how an \ncp\ designs its content distribution and traffic engineering strategy. 

%For instance, content $k \in K$ could represent either an on-demand video that can be viewed or a file that can be downloaded by the end-user.

%We model the network as  a set of nodes ($V$) and directed links ($E$) with capacity $C_e$ for link $e$. A node is equivalent to a PoP in an ISP network.  To provide CDN services through its network, the Network-CDN has storage and servers co-located at all the PoPs in the network. The storage available at node $i \in V$ is denoted by $D_i$. We assume that the Network-CDN has adequate compute resources available at each location to serve content.


A {\em content matrix} (CM) specifies the demand for each content at each node. An entry in this matrix, $T_{ik}, i\in V, k\in K$, denotes the demand (in bits/second) for content $k$ at node $i$. 
\textcolor{blue}{Each entry in CM is the average content demand over a few hours during the day, e.g. 8 AM - 12 noon.}


%CM is assumed to be measured by the \ncp\ a priori over a coarse-grained interval, e.g., the previous day.  The infrastructure required for this measurement is comparable to what ISPs have in place to monitor traffic matrices today.


Origin servers, owned and maintained by the NCDN's content providers, initially store all content published by content providers. We model origin servers using a single virtual origin node $o$ external to the \ncp\ that can be reached via a set of exit nodes $X \subset V$ in the \ncp.
% (Figure~\ref{fig:NCDNArch}). 
Since we are not concerned with traffic engineering links outside the \ncp, we model the edges $(x, o)$, for all $x \in X$, as having infinite capacity. \textcolor{blue}{The origin servers are assumed to have no limits on storage and throughput.} The virtual origin node $o$  always maintains a copy of all the requested content. However, a request for a content is served from the virtual origin node only if no copy of the content is stored at any node $i \in V$. In this case, the request is assumed to be routed to the virtual origin  via the exit node closest to the node where the request was made (in keeping with the commonly practiced {\em early-exit} or {\em hot potato} routing policy).



%ISP networks carry transit traffic in addition to NCDN traffic, which can be represented as a transit traffic matrix (TTM).  Each entry in the TTM contains the volume of transit traffic between two PoPs in the network. 



\subsection{Problem Statement}

We seeks to minimize the energy consumption of backbone routers and servers in an \ncp, while ensuring that maximum link utilization stays below the specified limit.  The energy consumption of routers can be reduced by routing traffic in a manner which serves the traffic using only a subset of links, which allows line cards connected to the inactive links to be turned off. The total energy consumption of servers is proportional to the number of servers turned in use across all PoPs.

%The total energy consumption of servers remains constant for a given workload under assumption of energy proportionality.



%
%
%\subsection{Cost Functions}
%\label{sec:costfunction}
%
%We evaluate NCDN-management strategies based on two cost functions. The
%first cost function is maximum link utilization (or MLU) which measures the effectiveness of traffic engineering in an NCDN. MLU is a widely used network cost function for traditional TE. 
%
%%the sum of latency of transfers in the network, which is equivalent
%The second cost function models user-perceived latency and is defined as  $\sum_{e \in E} X_e$, where $X_e$ is the product of traffic on link $e$ and its link latency $L(e)$.  The latency of a link $L(e)$ is the sum of  a fixed propagation delay and a variable utilization dependent delay. For a unit flow, link latency is defined as $ L_e(u_e) = p_e (1 + f(u_e))$, where $p_e$ is the propagation delay of edge $e$, $u_e$ is its link utilization, and $f(u)$ is a piecewise-linear convex function.
%
%% defined as $f(0) = 0$ and its first derivatives:
%
%\begin{eqnarray}
%  f'(u)  =  \begin{cases}  0 &  u \leq 0.6,  \notag \\
%  2 & 0.6 < u \leq 0.7,  \notag \\
%    4 & 0.7 < u \leq 0.8,  \notag \\
%      16 & 0.8 < u \leq 0.9,  \notag \\
%      128 & 0.9 < u \leq 1.0,  \notag \\      
%  2048 &  1  < u, \end{cases} \end{eqnarray}
%  
%This cost function is similar to that used by Fortz and Thorup \cite{fortz2000internet}. At small link utilizations ( $< 0.6$), link latency is determined largely by propagation delay hence $f$ is zero. At higher link utilizations (0.9 and above) an increase in queuing delay and delay caused by retransmissions significantly increase the effective link latency. The utilization-dependent delay is modeled as proportional to propagation delay as the impact of (TCP-like) retransmissions is more on paths with longer links.  Since $L_e$ is convex, a set of linear constraints can be written to constraint the value of $X_e$  (as in \cite{fortz2000internet}). %We omit these constraints here due to lack of space.

%The latency cost function is  $\sum_{e \in E} X_e$ where $X_e$ is the product of traffic on link $e$ and its link latency, $L(e)$. 

\subsection{Optimal Strategy as MIP}
\label{sec:linearprograms}




%The problem of placment-routing-redirection by \ncp s ({\em \ncp\ problem} for short) seeks to compute a placement and routing that minimizes the MLU and satisfies the demands specified by  the content matrix while respecting link capacity and storage constraints. This optimization goal can be formulated as a mixed integer program (MIP). Unlike the traditional traffic engineering problem that can be formulated as a multi-commodity flow problem and solved using a linear program, the \ncp\ problem needs to make binary placement decisions, i.e., whether or not to place a content at a PoP, and then route the demand accordingly. 

We present here a joint optimization strategy for minimizing backbone router and server energy use in an NCDN as a MIP. 
This formulation takes as input a content matrix, i.e., the demand for each content at each network point-of-presence (PoP), and  computes content placement, request redirection and routing that to minimize the energy use of backbone NCDN routers while respecting network and server resource constraints.
The decision variables for this problem are listed in  Table 1. The MIP to minimize an backbone router energy is as follows:
\begin{equation}
\min  \sum_{e \in E} 2 y_e L_e   + \sum_{i \in V} w_i R + \sum_{i \in V} v_i P 
\end{equation}
\begin{equation*}
\mbox{subject to}
\end{equation*}
\begin{eqnarray}
\sum_{j \in V }  t_{ijk} +   t_{iok} &=& T_{ik}, \quad \forall k \in K, i \in V\\
\sum_{k \in K} t_{ijk} &=& f_{ij} ,  \ \forall j \in V-X, i \in V\\
\sum_{k \in K} t_{ijk} + \sum_{k \in K} \delta_{ij}  t_{iok} &=& f_{ij} ,  \ \forall j \in X, i \in V
\end{eqnarray}
where $\delta_{ij}$ is 1 if $j$ is the closest exit node to $i$ and 0 otherwise. Note that $\delta_{ij}$ is not a variable but a constant that is determined by the topology of the network, and hence constraint (4) is linear.
\begin{eqnarray}
 \sum_{p \in P(l)} f_{ijp} - \sum_{q \in Q(l)}  f_{ijq}  =  \begin{cases}  f_{ij} & \text{if } l=i,  \notag \\
   -f_{ij}  &   \text{if } l=j,  \notag \\
0 & \text{otherwise}, \end{cases}\\
\forall i,j,l \in V
\end{eqnarray}
where $P(l)$ and $Q(l)$ respectively denote the set of outgoing and incoming links at node $l$.
\begin{eqnarray}
 \sum_{i \in V, j \in V} f_{ije} &\leq& \alpha \times C_e, \quad \forall e \in E\\
 \sum_{k \in K}  x_{ik}S_k &\leq& v_i D \times A_i , \quad \forall i \in V\\
x_{ok} &=& 1, \quad \forall  k \in K \\
\sum_{i \in V}  x_{ik} &\geq& z_k, \quad \forall k \in K 
\end{eqnarray}
 \begin{eqnarray}
x_{ik} &\leq& z_k, \quad \forall k \in K, i \in V  
\end{eqnarray}
\vspace{-0.25in}
\begin{eqnarray}
t_{ijk} &\leq&  x_{jk} T_{ik},  \quad \forall k \in K,  i \in V, j \in V \cup \{o\}\\
t_{iok} &\leq& T_{ik}(1 - z_k), \quad  \forall   k \in K 
\end{eqnarray}
\vspace{-0.25in}

%Ramesh: Hardwired constraint numbers. Dangerous but will leave for now.


The constraints have the following rationale.
Constraint (2) specifies that the total traffic demand at each node for each content must be satisfied.
Constraints (3) and (4) specify that the total traffic from source $j$ to sink $i$ is the sum over all content $k$ of the traffic from $j$ to $i$ for $k$.
Constraint (5) specifies that the volume of a flow coming in must equal that going out at each node other than the source or the sink.
Constraint (6) specifies that the total flow on a link is at most $\alpha$ times capacity.
Constraint (7) specifies that the total size of all content stored at a node must be less than its disk capacity.
Constraint (8) specifies that all content is placed at the virtual origin node $o$.
Constraints (9) and (10) specify that at least one copy of content $k$ is placed within the network if $z_k = 1$, otherwise $z_k = 0$ and no copies of $k$ are placed at any node. Constraint (11) specifies that the flow from a source to a sink for some content should be zero if the content is not placed at the source (i.e., when $x_{jk} = 0$), and the flow should be at most the demand if the content is placed at the source  (i.e., when $x_{jk} = 1$).
Constraint (12) specifies that if some content is placed within the network, the traffic from the origin for that content must be zero. 


% constraint 1: if a link is off, it cannot transmit any traffic 

Servers at a PoP cannot send traffic at a speed greater than their maximum throughput of servers that are turned on at a node.
\begin{eqnarray}
\sum_{i \in V} \sum_{k \in K} t_{ijk}  \leq  v_i N \times A_j,\quad\forall j \in V
\end{eqnarray}

If a link is inactive, it cannot transmit any traffic.
\begin{eqnarray}
\sum_{i \in V, j \in V} f_{ije} &\leq& y_e \times C_e, \quad \forall e \in E
\end{eqnarray}


If the router chassis is off, all links connected to the router are inactive.
\begin{eqnarray}
y_e \leq w_i, \quad \forall i \in V,  e\in \textit{set of incoming and outgoing links at node } i
\end{eqnarray}


% constraint 3: if chassis is off, then all links are off as well

If all links are inactive, router chassis is turned off as well.
\begin{eqnarray}
w_i \leq  \sum_e y_e,  \quad\forall i \in V,  e\in \textit{set of incoming and outgoing links at node } i
\end{eqnarray}

Variables $w_i$ and $y_e$ are binary.
\begin{eqnarray*}
w_i, y_e \in \{0, 1\}
\end{eqnarray*}

\begin{eqnarray*}
x_{jk}, \ z_k &\in& \{0,1\}, \quad \forall j \in V, k \in K\\
y_{e}  &\in& \{0,1\}, \quad \forall e \in E\\
w_{i}  &\in& \{0,1\}, \quad \forall i \in V\\
v_i &\in& \{0, 1, .., A_i \} \quad \forall i \in V \\
f_{ije}, \ t_{ijk}, \ t_{iok} &\geq& 0, \quad \forall  i,j \in V, e \in E, k\in K
\end{eqnarray*}


\subsection{Heuristic strategies}

\subsection{Model parameters selection}


\textbf{Network topology}: We will use network topology map of Tier-1 ISPs available in the RocketFuel dataset, CAIDA router-level Internet topology (http://www.caida.org/data/active/internet-topology-data-kit/index.xml), as well as actual topology maps obtained from ISPs themselves. 

\textbf{Origin locations}: We select a small number (say 3) of geo-distributed origin locations, each of which accessible via one of the AS nodes.


\textbf{Number of servers}:  Large CDNs today have tens of thousands of servers deployed globally. An NCDN caters only to users in its own network, so we expect it to deploy a few hundred to a few thousands of servers depending on the size of the NCDN. The distribution of servers across nodes is proportional to the request rate at the PoP.


\textbf{Server storage, throughput, and power consumption}: Each server is assumed to a have 1 Gb connection to the next hop switch/router.  The actual throughput could be less if CPU / disk is the bottleneck. We can expect per server to have few hundred Mbps of throughput (N). Each server has a D = 1 TB local disk storage.  Each server consumes P = 100 W when it is turned on and zero when turned off. 

\textbf{Router energy parameters}: Energy consumption of chassis is R = 600 W, and line cards consume 100 - 200 W depending on their capacity.

\textbf{Performance constraint:} Upper bound on MLU is $\alpha = 0.75$ as both queuing delay and loss rates are minimal until this level of utilization. 

\textbf{Content demand:} We will use combined on-demand video traces of all Akamai customers for the experiment. The traces may be scaled so that the aggregate content demand is commensurate with the total server capacity in an NCDN.


% 300 Mbps / server, 1000 servers = 300 Gbps server throughput 
% Cache hit rate = 90%, network traffic = 30 Gbps
% Network bandwidth is of the order of tens of Gbps, so utilization level should be around 50 %. 

% with 1000 servers energy = 100 KW (100 W/server)
% with 50 routers energy = 50 KW. (1 KW /router)











% constraint 4: a server cannot send more traffic than its specified capacity






%Updating the content placement itself generates traffic and impacts the link utilization in the network. For ease of exposition, we have deferred a formal description of the corresponding constraints to our tech report \cite{techreport}.  Finally, a simple extension to this MIP presented in the tech report \cite{techreport}  jointly optimizes routing given a TTM as well a CM.  We have presented a CM-only formulation here as our findings (in $\S$\ref{sec:eval}) show that a joint optimization of the CM and TTM is not useful for NCDNs.

%\subsubsection{ISP Transit Traffic Optimization}
%
%\subsection{Computational Hardness}
%\optloc\ is the decision version of the NCDN problem. The proofs for these theorems are presented in Appendix  \ref{sec:npcreduction}.
%%We show that the NCDN-problem is NP-Complete and is inapproximable to within an exponential factor.  
%
%{{\sc Theorem 1.}} {\em \optloc\ is NP-Complete even in the special case where all objects have unit size, and all demands, link capacities, and storage capacities have binary values.}
%
%%Further, NCDN is inapproximable within an exponential factor unless P = NP. 
%
%{{\sc Corollary 1.}} {\em  \optloc\ is inapproximable to within a constant factor  unless P = NP.}
%



%We prove that the NCDN problem is NP-Complete. The proof is included in \cite{techreport}.





%{{\sc Theorem 1}} {\em The \ncp\ problem is NP-complete.}

% In general, solving an MIP for large problem sizes is computationally challenging. We formally prove that the \ncp\ problem is intrinsically computationally hard. (Note that being able to formulate a problem as an MIP does not imply that the problem is hard enough to warrant it.). The proof of the theorem below proceeds via a reduction from the subset-sum problem and is described in the Appendix.

%
%{{\sc Theorem 1}} {\em The \ncp\ problem is NP-complete.}

\eat
{
\subsection{Partial Optimization Strategies}
\label{sec:mipinvcap}
As one of our goals is to analyze the relative importance of optimizing placement and routing, we describe how the above MIP can be modified to calculate optimal content placement with fixed routing, e.g., InvCap. The key idea is to write the value of $f_{ije}$, 

 the values of variables  as per the given (fixed) routing and optimize for the placement and redirection variables. 
  The complete description of the modification 

Assume that the given (fixed) routing strategy is specified in terms of the variables $ r_{ije}, 0 \leq r_{ije} \leq 1$, for $i,j \in V$ and $e \in E$, that represents the fraction of flow $f_{ij}$ that flows on the link $e \in E$. Assuming that the specified routing is valid, constraint (5) of the MIP that enforces the conservation of flow is no longer needed and can be removed. Further, all variables $f_{ije}$ can be replaced in the MIP with by $r_{ije} f_{ij}$, resulting in a reformulation of the LHS of constraint (6) of the MIP.

For a \unplanned\ placement strategy such as, LRU, we model route optimization as a multi-commodity flow problem (which is an linear program) identical to  the traditional traffic engineering problem \cite{fortz2000internet}. We assume that the \ncp\ measures the traffic matrix over the immediately preceding monitoring interval and computes routes that optimize the MLU for that matrix. The matrix incorporates the effect of the \unplanned\ placement strategy and the implicit assumption is that the content demand and \unplanned\ placement strategy result in a traffic matrix that does not change dramatically from one monitoring interval to the next---an assumption that also underlies traffic engineering as practiced by ISPs today. 
}
%Therefore, we remove Equations \ref{eq:flow1},  \ref{eq:flow2}, \ref{eq:disk5}, \ref{eq:disk6}, \ref{eq:flow7} and substitute the values of the variables $y_{ik}$ in Equation \ref{eq:flow3} from the MIP in Section \ref{sec:optmip}.




%\noindent\textbf{Solving the joint optimal flow splits and routing problem for a fixed placement}
%\label{sec:optflowsplitsrouting}
%This problem is a linear program as well since the variables $x_{jk}$ are known. We formulate the linear program for this problem by removing Equations \ref{eq:disk5} and \ref{eq:disk6} and replacing the value of the variable $x_{jk}$ in Equation \ref{eq:flow7} from the MIP in Section \ref{sec:optmip}.
%
%
%
%\noindent\textbf{Solving the optimal flow splits problem for a fixed routing and a fixed placement}
%\label{sec:optflowsplits}
%The linear program for this problem can be obtained by by removing Equations \ref{eq:flow3}, \ref{eq:disk5} \ref{eq:disk6}, including Equation \ref{eq:fixedrouting} and replacing the value of the variable $x_{jk}$ in Equation \ref{eq:flow7} from the MIP in Section \ref{sec:optmip}.


%\subsubsection{Traffic Engineering with Heuristic Static Placement}
%\label{sec:heuristic}
%




% and considered requests belonging to the sample . 
% 
%
% 
%Some of the traces we experimented with contained tens of thousands of content. In these cases, we sampled a fraction of content from the trace. Then, we selected only those entries from the trace which belong to the sampled set of objects and performed our experiment with the sample trace.  
%
%Some of the traces we experimented with contained tens of thousands of content With the above two techniques, we needed to optimize for less than 5000 content in our experiments.   



%In Appendix \ref{sec:mipinvcap}, we describe two variations of the MIP presented above which optimize routing for a fixed placement strategy and optimize placement for a fixed routing strategy (e.g. InvCap) respectively.

%As solving the MIP for very large problem scenarios is computationally infeasible, we use standard approximation techniques such as LP-relaxation and sampling which we describe in Appendix \ref{sec:approx}.


\eat {
\subsection{Optimal TE for Network-CDN Model}

Solving the \ncp\ problem optimally is NP-Complete. We present a reduction from the subset-sum problem in the Appendix  \ref{sec:npcreduction}.

In this section, we present a MIP formulation which optimally solves the TE problem for the model presented above. The network cost function we optimize is the maximum link utilization (MLU) of the network. 
}
 
 
 \eat{
\subsubsection{Traffic Engineering Variables}

The optimization objective, i.e., MLU, can be written in terms of following variables and the  model parameters discussed above. 

\noindent\emph{Content Location Variables ($x_{jk}, z_{k}$):} These variables decide the nodes at which content is placed in the network. Both $x_{jk}$ and $z_k$ are binary variables.
The variables $x_{jk}$ determines whether a content $k \in K$  is stored at the node $j \in (V\cup O).$ Note that $j$ could either be a origin node or a node in the network. Since all content is present at all origin nodes, $x_{jk} = 1$ if $j \in O$. The variable $z_k$ indicates whether a content is present in the network or not. If $z_k = 1$, then the content is present in the network, otherwise it is present only at origin nodes.

\noindent\emph{Redirection Variables ($t_{ijk}$):}  If a request for a content $k \in K$, arrives at node $i$, then what fraction of requests should be sent to the node $j$.

\noindent\emph{Routing Variables ($f_{ice}, f_{ij}$):}  Variable  $f_{ij}$ denotes the total traffic in bits/sec from node $j$ to node $i$, $j,i \in V$ and variable $f_{ije}$  denotes the traffic from node $j$ to node $i$ which goes on link $e$.  The ratio of variable $f_{ice}$ to $f_{ij}$  defines the routing in the network. 

%Routing in the network can be uniquely defined if we know the fraction of traffic between all pairs of nodes ($i$,$j)  which crosses a given link $e$. Based on this definition, the ratio of variable $f_{ice}$ to $f_{ij}$ represents the routing in the network. Variable  $f_{ij}$ denotes the total traffic in bits/sec from node $j$ to node $i$, $j,i \in V$ and variable $f_{ije}$  denotes the traffic from node $j$ to node $i$ which goes on link $e$. 



\subsubsection{MIP Formulation}

\label{sec:linearprograms}

In this section, we prove that optimally solving the joint placement and routing problem is NP-Complete. We present our mixed integer programming (MIP) formulation for this problem. Next, we describe MIPs/linear programs for other trafic engineering schemes which optimize placement and/or routing.


An MIP formulation for this problem is presented in Section \ref{sec:optmip}. This MIP formulation is useful if the set of content, demand for a content at each PoP in the network is known. The  storage at each PoP and capacity of each link is also necessary to solve the MIP. 
The solution to the MIP provides the routing in the network. For each content, the solution tells us which PoP/s will each content be stored and  requests for a content from a PoP will be redirected to which other PoP/s if the content is not available at that PoP.




\label{sec:optmip}



For our network model described in Section \ref{sec:model}, we have to solve the following mixed integer program.

\noindent\textbf{Minimize: } $\alpha$  (MLU for the network)

\noindent\textbf{Subject to:}

\begin{equation}\label{eq:flow1}
\mbox{$\min$ } \alpha
\end{equation}
subject to 
\begin{eqnarray}\label{eq:flow1}
 \sum_{j \in V }  t_{ijk} +  \sum_{o \in O}  t_{iok} &=& T_{ik}, \ \ \forall k \in K, i \in V
\end{eqnarray}
\begin{eqnarray}\label{eq:flow201}
\sum_{k \in K} t_{ijk} &=& f_{ij} , \ \ \forall j \in V - X, i \in V\\
\sum_{k \in K} t_{ijk} + \sum_{k \in K} t_{iok} &=& f_{ij} , \ \ \forall j \in X, i \in V
\end{eqnarray}
where $o$ is the virtual origin node adjacent to $j$.

%\begin{equation}\label{eq:flow2}
%\sum_{k \in K} t_{ijk} = f_{ij} , \forall j \in (V - X), i \in V
%\end{equation}
\begin{eqnarray}
 \sum_{p \in P(l)} f_{ijp} - \sum_{q \in Q(l)}  f_{ijq}  =  \begin{cases}  f_{ij} & \text{if } l=i,  \notag \\
   -f_{ij}  &   \text{if } l=j,  \notag \\
0 & \text{otherwise}, \end{cases}\\
\forall i,j,l \in V  \label{eq:flow3}
%\\ && \ \ \forall i,j,l \in V  \label{eq:flow3}
\end{eqnarray}
%\[\mbox{if } l = i, d = f_{ij} ;   \mbox{ if } l = j, d = - f_{ij};  \mbox{otherwise } d = 0.\]
%where $d$ is 1 if $l=i$, is $-1$ if $l=j$, and is 0 otherwise.
%%%%%%%%%%
\begin{eqnarray}
 \sum_{i \in V, j \in V} f_{ije} &\leq& \alpha \times C_e, \quad \forall e \in E \label{eq:mlu4} \\
 \sum_{k \in K}  x_{ik}S_k &\leq& D_i , \quad \forall i \in V \label{eq:disk5}
\end{eqnarray}
%%%%%%%%%%
\begin{eqnarray}
x_{ok} &=& 1, \quad \forall o \in O,   k \in K \label{eq:disk501}\\
\sum_{i \in V}  x_{ik} &\geq& z_k, \quad \forall k \in K  \label{eq:disk6}\\
x_{ik} &\leq& z_k, \quad \forall k \in K, i \in V  \label{eq:disk601}
\end{eqnarray}
%%%%%%%
\begin{eqnarray}
0 \leq t_{ijk} \ \leq\  x_{jk} T_{ik}, & \forall k \in K,  i \in V, j \in V \cup O   \label{eq:flow7}\\
t_{iok} < T_{ik}(1 - z_k), & \forall o \in O,  k \in K   \label{eq:flow8}
\end{eqnarray}

\begin{eqnarray*}
x_{jk} &=& \{0,1\}, \quad \forall j \in V, k \in K\\
z_{k} &=& \{0,1 \}, \forall k \in K\\
f_{ije} &\geq& 0, \quad \forall  i,j \in V, e \in E
\end{eqnarray*}



The constraints above have the following meanings.

\noindent(2) The total traffic demand at each node for each content must be satisfied.

\noindent(3, 4) The total traffic from source $j$ to sink $i$ is the sum over all content $k$ of the traffic from $j$ to $i$ for $k$.

\noindent(5) The volume of a flow coming in must equal that going out at each node other than the source or sink.

\noindent(6) The total flow on a link is at most $\alpha$ times capacity.

\noindent(7) The total size of all content stored at a node must be less than its disk capacity.

\noindent(8) All content is placed at each origin node.

\noindent(9, 10) At least one copy of content $k$ is placed within the network if $z_k = 1$, otherwise at most $z_k \ (=0)$ copies are placed at any PoP.

\noindent(11) The flow from a source to a sink for some content should be zero if the content is not placed at the source.

\noindent(12) If some content is placed within the network, the traffic from the origin for that content must be zero.


\eat {
\begin{enumerate}
\item 
The total traffic demand at each node for each content must be satisfied. 

\item 
The total traffic $y_{ik}$ from source $k$  to sink $i$ is composed of traffic  for each content $j$, $y_{ijk}$.





\item 

Treat each flow $y_{ik}$ as an independent flow which can be routed independently. The flow conservation constraints for routing flow  $y_{ik}$ are as follows:


%(a)  if $l = k$ then $d = -y_{ik}$, (b) if $q = i$ then, $d = y_{ik}$ , (c) otherwise $d = 0$. 

In addition $0 \leq h^{qp}_{ik} \leq f_{ik}$ as mentioned above.

\item
(Total flow on each link) $\leq $ $\alpha $ (capacity of link). For each link $e_{pq}$, 

\item
The total disk capacity at each node $i$ must not exceed total size of files at that node.


\item
Each content must have at least one copy in the network.



%\item

%The total traffic $y_{ik}$ from  from source $k$  to sink $i$ must be less than the total traffic demand at sink $i$ for the files stored at source $k$.

%\[ y_{ik} \leq \sum_{j} x_{jk} f_{ij} \]

\item
Flow from node $i$  to node $k$ for content $j$, i.e., $y_{ijk}$ must be zero if the content is unavailable at that node.
 



\item
Accounting for cost of content placement update
\item
Accounting for traffic from origin
\end{enumerate}
}
}



%CPLEX solver. 
%
%To this end, we sort all content in decreasing order of popularity and discard the least popular 1\% of the content. 
%
%This simple technique reduces the number of objects by 50\%  or more in most of our traces. 
%
%select a prefix of this sequence to be placed within the network. We discard the least popular 
%We find that even discarding the tail of unpopular content accessed at most twice a day (the placement recomputation period) makes the problem size tractable for the experimental scenarios considered in this paper. \tbd{Arun: Do we use this technique for approximation?} Another alternative is to use a ``potential method'' technique such as that used in  \cite{Applegate2010}, but we defer that to future work.

%A second assumption we make for computational tractability is that the number of unique content is the network is of the order of few hundreds. This is because it becomes computationally intractable to optimally solve the linear program when the number of unique content is in thousands. It should be possible to reduce the time to solve the linear program using "potential method" technique used in \cite{Applegate2010}, but we have not investigated that approach.


%Solving the \ncp-problem optimally is computationally challenging because it is a MIP. The variables which determine content placement ($x_{jk}$'s) are the only integer variables in the MIP. To solve this problem for our input sizes, we adopt a two step process. In the first step, we solve an \emph{LP} assuming $x_{jk}$ can take value between 0 and 1. In the second step, we solve the original MIP, but we set those  $x_{jk}$ values to $0$ which are equal to $0$ in the LP solution. This is a common approximation technique used to solve MIP problems \cite{ILPapprox}. 

%We observe that the above approximation not only eliminates up to 80\% of integers variables in most cases.


\eat
{


\subsection{Content Chunking}

A widely used technique to improve efficiency of content delivery is content chunking. For example HTTP supports caching of partially downloaded content \cite{rfc2616}, BitTorrent distributes content in small chunks \cite{bittorrentprotocol}.

A \unplanned\ placement improves with content chunking due to two reasons: (1) A chunk is downloaded much before the complete file is downloaded, and hence it can be uploaded to other caches sooner. (2) Content chunking enables caches to store a partially downloaded content if the user aborts the download before completion. Without content chunking support, partially downloaded content is useless and hence must be discarded from cache. The alternative is to download the remaining fraction of content to the cache even though the user has aborted the download. Both these choices lead to a wastage of network bandwidth and the later also cause unnecessary cache evictions.  Splitting large content into smaller chunks significantly reduces network bandwidth wastage, reduces avoidable cache evictions and improves network cost.

A \unplanned\ placement benefits from content chunking for the following reasons: (1) Chunks of a content often differ in popularity.  For example, a chunk consisting of the first few minutes of video is likely to be viewed more often than the complete video, often because of loss of user's interest \cite{youtubestudy}. In this case, storing popular chunks at more PoPs is better.  (2) Chunking makes it possible to store more content at each PoP. For example, a large file may not  fit at any PoP due to storage constraints,  but its chunks can stored across a set of PoPs. 
(3) Splitting a content across multiple locations spreads the traffic for that content over more links and can potentially reduce network cost. It is easy to construct examples where storing two halves of a content at two nodes leads to a lower MLU than storing complete content at a single node.


\subsection{Link Load Aware Request Redirection}

In order to reduce network cost, NCDN can use link load information which can be collected using SNMP protocol to make request redirection decisions. A simple approach which we implement is following:  if a user's request results in a cache miss at the local PoP, we fetch content from other PoPs  which may have the content available. If multiple PoPs have the content, we choose that PoP for which the utilization of the most utilized link  along the path from that PoP to the local PoP is the least. We break ties based on hop count distance and then randomly.

}


%Based on above MIP, content chunking enables $x_{jk}$ variables to take fractional values in addition to binary values, which will likely result in a smaller value of objective function.

%As a concrete example,  if no single PoP has enough space to accommodate a large content  but its chunks can stored across a set of PoPs.


%We split any video longer than 5 minutes into chunks of 5 minute duration, except for the last chunk which could be of a smaller duration. For the downloads trace, we split content into chunks of size 50 MB, except for the last chunk. Our experiments with content chunking use the demand of each chunk, instead of the demand for the original content to calculate demand aware placement; demand oblivious placement (caching) treats each chunk as a distinct content to be either cached or evicted.  

