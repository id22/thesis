\begin{abstract}



%Traffic engineering (TE) is a well-studied network optimization problem but its effect on the end-user application performance has not received the same attention. We take the first step in this direction by presenting a comparison of common TE schemes based on application performance metrics. Further, the effect of application-level adaptation on TE is not well understood. We analyze the effect of application adaptation in the form of \emph{location diversity}, i.e., the ability to download content from many locations, on TE schemes in terms of their tolerance to accommodate surges in traffic demand.
%Using real traffic matrices and topologies from three ISPs, we conduct very large-scale experiments simulating ISP traffic as an aggregate of a large number of TCP flows. Our empirical approach reveals that that link utilization based TE metrics such as MLU are  poor predictors of application performance for typical Internet workloads; they also fail to capture the surge tolerance of TE schemes under the effect of location diversity. Surprisingly, we find that all TE schemes show nearly identical application performance for Internet traffic workloads. All schemes even achieve near-optimal surge tolerance  with only a small degree of location diversity (2-4 locations). Our findings call into question the value of TE as practiced today, and compel us to significantly rethink the TE problem in the light of application adaptation.

Traffic engineering (TE) has been long studied as a network optimization problem, but its impact on user-perceived application performance has received little attention. Our paper takes a first step to address this disparity. Using real traffic matrices and topologies from three ISPs, we conduct very large-scale experiments simulating ISP traffic as an aggregate of a large number of TCP flows. Our application-centric, empirical approach yields two rather unexpected findings. First, link utilization metrics, and MLU in particular, are poor predictors of application performance. Despite significant differences in MLU, all TE schemes and even a static shortest-path routing scheme achieve nearly identical application performance. Second, application adaptation in the form of {\em location diversity}, i.e., the ability to download content from multiple potential locations, blurs differences in the achieved capacity of TE schemes. Even the ability to download from just 2--4 locations enables all TE schemes to achieve near-optimal capacity, and even static routing to be within 30\% of optimal. Our findings call into question the value of TE as practiced today, and compel us to significantly rethink the TE problem in the light of application adaptation.

%In this work, we revisit the traffic engineering (TE) problem focusing on user-perceived application performance, an aspect that has largely been ignored in prior work. Using real traffic matrices and topologies from three ISPs, we conduct very large-scale experiments simulating ISP traffic as an aggregate of a large number of TCP flows. Our application-centric, empirical approach yields two rather unexpected findings. First, link utilization metrics, and MLU in particular, are poor predictors of application performance. Despite significant differences in MLU, all TE schemes and even a static shortest-path routing scheme achieve nearly identical application performance. Second, application adaptation in the form of location diversity, i.e., the ability to download content from multiple potential locations, significantly impacts TE. Even the ability to download from just 2--4 locations enables all TE schemes to achieve near-optimal capacity, and static routing to be within 30\% of optimal. Our findings call into question the value of TE as practiced today, and compel us to significantly rethink the TE problem in the light of application adaptation.



%In this paper we focus on two aspects of traffic engineering (TE) which have largely been ignored in prior work: (1) the effect of TE on application performance, and (2) the effect of application adaptation on TE. Using large scale ns-2 simulations of traffic matrix data from three ISPs, we find that there is little difference in application performance for all TE schemes. Further, we find that link utilization based metrics such as maximum link utlization (MLU) are poor predictors of application performance.
%We study the effect of application adaptation in the form of location diversity, i.e, the ability to download from multiple locations. We show that location diversity increases the capacity of the network for all TE schemes. An important consequence of location diversity is that all TE schemes achieve near-optimal capacity even with a small degree of location diversity (2-4 locations).



%Traffic engineering methods optimize for maximum link utilization (MLU) or other synthetic cost functions based on link utilization. It is unclear how these cost functions relate to the application performance for internet traffic matrices. Using large scale ns-2 simulations of traffic matrix data from 3 ISPs, we find that traffic engineering makes little difference to file download rates using TCP for traffic matrices at Internet load. We also find that traffic engineering approaches such as COPE\cite{COPE1} which engineer for unpredictable spikes in traffic increases delay and hurts TCP throughput.

%Using location diveristy application adaptation in the form of location diversity, i.e., the ability to download data from multiple sources in the network. We show this adaptation can increase the capacity of ISP networks by approximately 1.4$\times$ over a network without any location diversity. A surprising consequence of adaptation behavior is that it nearly vanishes difference in capacity among traffic engineering schemes. Even shortest path routing \optwt can achieve nearly same capacity as optimal traffic engineering even with small degree of location diversity( 2--4 locations).




%Traffic engineering methods optimize for maximum link utilization (MLU) or other synthetic cost functions based on link utilization. It is unclear how these cost functions relate to the application performance for internet traffic matrices. Using large scale ns-2 simulations of traffic matrix data from 3 ISPs, we find that traffic engineering makes little difference to file download rates using TCP for traffic matrices at Internet load. We also find that traffic engineering approaches such as COPE\cite{COPE1} which engineer for unpredictable spikes in traffic increases delay and hurts TCP throughput.
%
%Application adaptation is  observed widely in the internet today in the form of P2P networks and CDNs which constantly adapt to the congestion in the network. We simulate application adaptation in the form of location diversity, i.e., the ability to download data from multiple sources in the network. We show this adaptation can increase the capacity of ISP networks by approximately 1.4$\times$ over a network without any location diversity. A surprising consequence of adaptation behavior is that it nearly vanishes difference in capacity among traffic engineering schemes. Even shortest path routing \optwt can achieve nearly same capacity as optimal traffic engineering even with small degree of location diversity( 2--4 locations).

\end{abstract}
