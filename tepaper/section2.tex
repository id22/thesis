\section{Engineering traffic with location diversity}
\label{sec:locdiv-background}

This section introduces location diversity and propose a new metric to quantify the capacity achieved by traffic engineering schemes with location diversity.


\subsection{Location diversity: Prevalence}

Location diversity, or the ability to download content from multiple potential locations, is widespread in the Internet today. Major commercial CDNs, e.g., Akamai \cite{Akamai}, Level-3 \cite{Level-3}, EdgeCast \cite{edgecast} etc., commonly replicate content at hundreds of locations and redirect users to the best server based on proximity or dynamic monitoring of server and network congestion \cite{akamai-detour}. Popular P2P applications such as BitTorrent \cite{bittorrentprotocol}, PPLive \cite{PPLive} download content simultaneously from many peers that are chosen based on a number of factors including network congestion. Other examples of location diversity include cloud computing infrastructure providers such as Google and Amazon with geographically distributed sites; content hosting services such as Carpathia \cite{Carpathia}, Rapidshare \cite{OneClickHosting}, etc.; mirrored websites such as SourceForge, Debian, etc. 


Although quantifying the extent of location diversity in today's Internet is difficult, back-of-the-envelope calculations based on existing measurement studies suggests that it is significant.  CDNs alone are estimated to account for 10\% of Internet traffic \cite{AtlasReport}. Major cloud computing and content hosting companies with location diversity contribute to a significant fraction of Internet traffic, e.g., Google (6\%), Comcast (3\%), RapidShare (5\%) and Carpathia (0.5\%), a trend that is projected to increase in the near future \cite{urlinternet,AtlasReport}. The fraction of P2P traffic in Internet was estimated to be between 18-60\% by different measurement studies in 2009.  


\subsection{Location diversity: Quantifying capacity}

How can we quantify the capacity achieved by a TE scheme in the presence of location diversity? In general, the capacity is a {\em region} that includes all of the traffic matrices that it can accommodate. However, quantifying the capacity of a TE scheme as a region may shed little light on its ability to tolerate typically encountered load spikes. Furthermore, it is cumbersome to compare TE schemes that achieve overlapping capacity regions. So, it is common to use a more concise metric such as the MLU to characterize the capacity with respect to a given traffic matrix. Intuitively, the inverse of the MLU serves as a metric of capacity, e.g., if a TE scheme achieves an MLU of 0.25 for a given matrix, then it can tolerate up to a 4$\times$ surge in the load represented by the matrix. Unfortunately, MLU is not a meaningful metric of capacity when application adaptation to location diversity  determines the traffic matrix (Chapter \ref{ch:te-background},  Section \ref{sec:bg-3node}).
  

With location diversity, the demand is best represented as a ``content matrix'' that specifies for each node and each content the traffic for that content at that node and the set of source locations from where that content can be downloaded. The traffic matrix corresponding to this demand depends upon the underlying routes and application behavior (e.g., how parallel TCP splits traffic across the download locations). Furthermore, scaling the demand does not simply scale the traffic matrix entries by the same factor. In general, it is difficult to predict how application behavior might change the traffic matrix for a projected surge in demand, as that change depends upon the underlying routes that in turn depend upon the original traffic matrix. Indeed, as the example in Chapter \ref{ch:te-background},  Section \ref{sec:bg-3node} shows, even if the demand is unchanged, the mere act of engineering routes can change the traffic matrix yielding a different MLU than expected.


\subsubsection{An empirical capacity measure}
\label{sec:SPFdefinition}
%\textbf{proposing a new metric}

We propose a new metric, {\em surge protection factor} (SPF), to quantify the capacity achieved by a TE scheme with respect to a traffic matrix. Let $E$ denote a TE scheme, $M$ the demand specified as a content matrix. When there is no location diversity, $M$ can be easily transformed to a unique traffic matrix $T(M)$. Let  MLU$(E,T(M))$ denote the MLU achieved by $E$ given the traffic matrix $T(M)$. In this case, SPF$(E,M)$ is simply the inverse of MLU$(E,T(M))$, i.e., the factor of increase in the demand that can be satisfied. However, in the case when there is location diversity, SPF$(E,M)$ is an {\em empirical} measure of the satisfiable increase in demand computed as follows. Let $kM$ denote the demand that scales each entry in $M$ by a factor $k>1$. Then, SPF$(E,M)$ is defined as the largest $k$ such that the routing computed by $E$ (for the matrix $T(M)$) can satisfy the demand $kM$.

Determining if an engineering scheme can satisfy a projected demand is difficult as it requires us to accurately model application adaptation to location diversity, so SPF is useful mainly as an empirically measured capacity metric. To this end, we describe our experimental setup next.
