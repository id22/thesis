\section{APPENDIX}
\label{sec:appendix}
\subsection{Linear program to calculate optimal routing for flows with multiple sources}
Network  $G = (V,E)$.

Nodes $V = \{ n_1, n_2, ... n_a  \}$

Links $ E = \{ e_{ij}\}$ for link between node $i$ and  $j$ and link capacity = $ C = \{ c_{ij}\}$.

Location diversity parameter $k$. This is the number of sources for each flow.

Flows $F = \{f_1, f_2, ... f_b\}$ 

Sources for flow $f_i$, $S_i = \{ s_{i1},s_{i2},....s_{ik} \}$. Destination for flow $f_i$ = $d_i$.

%Routing configured in the network: 

For the above network and traffic demands, we have to solve the following linear program.


Minimize:  $\alpha$, the maximum link utilization.

Subject to:
\begin{enumerate}
\item 
Each flow demand must be routed. For each flow $f_i$,
\[ f_i = \sum_{i = 1 to k }  f_{is_{ij}}, 0 \leq  f_{is_{ij}} \leq f_i \]
where each $f_{is_{ij}}$ amount of flow $f_i$ routed from source node $s_{ij}$ to destination node $t_i$. 

\item 
Treat each flow as an independent flow which can be routed independently. For flow $f_{is_{ij}}$, . The constraints for routing this flow are :

For each node $q$,
\[ \sum_{r \in outgoing(q)} h_{qr}^{is_{ij}} - \sum_{p \in incoming(q)} h_{pq}^{is_{ij}} = d\]
$d = -f_{is_{ij}}$ if $q = s_{ij}$, $d = f_{is_{ij}}$ if $q = t_i$, otherwise $d = 0$. In addition $0 \leq h_{ij}^{st} \leq f_{is_{ij}}$ as mentioned above.

\item
(Total flow on each link) $\leq $ $\alpha $ (capacity of link). For each link $e_{pq}$, 
\[ \sum h_{pq}^{is_{ij}} \leq \alpha \times c_{pq} \]
for each link.

\end{enumerate}

The above formulation can provide us the global optimal solution for flows having multiple sources but a single destination.

\subsection{Linear program to calculate optimal MLU for a given routing}
The variables defined above are reused for this program. We define a new set of variables for the routing to be configured.

\[R = \{r_{ij}^{st}\}, 0 \leq r_{ij}^{st} \leq 1\]
$r_{ij}^{st}$ is the fraction of flow from source node $s$ to destination node $t$ which passes on the link between node $i$ and $j$. 
The routing defined must should obey following constraints. For each node $j$,
\[ \sum_{k \in outgoing(j)} r_{jk}^{st} - \sum_{i \in incoming(j)} r_{ij}^{st} = d\]
$d = -1$ if $j = s$, $d = 1$ if $j = t$, otherwise $d = 0$. In addition $0 \leq r_{ij}^{st} \leq 1$ as mentioned above. The linear program is as follows:

Minimize:  $\alpha$, the maximum link utilization.
Subject to:
\begin{enumerate}
\item \item 
Each flow must be routed: For each flow $f_i$,
\[ f_i = \sum_{j = 1 \mbox{ to }k }  f_{is_{ij}}, 0 \leq  f_{is_{ij}} \leq f_i \]
where each $f_{is_{ij}}$ amount of flow $f_i$ routed from source node $s_{ij}$ to destination node $t_i$. 


\item 
Total flow between a source and destination is defined as $g_{st}$
\[ g_{st} = \Sigma f_{is_{ij}},   \forall i,j : s_{ij} = s , t_i = t \]

\item
(Total flow on each link) $\leq $ $\alpha $ (capacity of link). For each link $e_{ij}$, 
\[ \sum g_st \times r_{ij}^{st} \leq \alpha \times c_{ij} \]
for each link.

\end{enumerate}

If the routing in the network is defined, i.e. $r_{ij}^{st}$ are constants, then above defition is a linear program. This formulation can be used to compute the capacity for \invcap{} or any other known routing scheme. When $r_{ij}^{st}$ are variables, the above program is not a linear program as defined above.

%For each $f_{is_{ij}} : s_{ij} = s , t_i = t$, let $h_{is_{ij}}^{pq} = f_{is_{ij}} \times r_{pq}^{st}$. Thus, $h_{is_{ij}}^{pq}$ is the part of flow  $f_{is_{ij}}$ which goes on link $e_{pq}$.Thus, we elide the variables $r_{ij}^{st}$ mentioned in equations above. Our new set of constraints are as follows.

%\[ \sum h_{is_{ij}}^{pq} \leq \alpha \times c_{pq} , s_{ij} = s , t_i = t\]
%Even after this rewriting, the problem isnt a linear program since there is no way to relate $h$'s to $f$'s.



\end{document}
