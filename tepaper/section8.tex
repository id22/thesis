\section{Conclusion}
\label{sec:conclusion}

Our comparison of TE schemes based on user-perceived metrics while accounting for application adaption to location diversity reveals unexpected results that challenge conventional wisdom in traffic engineering. 
We find that link utilization, the most widely used metric to evaluate TE, is a poor predictor of user-perceived performance. 
Under typical Internet load conditions, all TE schemes and even static routing achieve nearly identical user-perceived performance despite achieving vastly different MLUs.  In fact, engineering link utilization in order to accommodate unexpected traffic spikes  can actually hurt common-case application performance. 
More intriguingly, we find that application adaptation to location diversity, or the ability to download content from multiple locations, eliminates differences in the achieved capacity of all TE schemes including ``optimal" TE. 
With location diversity, even static routing achieves a capacity that is at most 30\% (and typically significantly less) worse than the optimal. Taken together, our findings suggest that it matters little which TE scheme is used at today's traffic levels as well as under reasonable projections of increased demand. 

%A provocative interpretation of our findings is that which TE scheme is used or whether TE is employed at all matters little for application performance under typical load conditions today, and matters little for application performance under reasonable projections of increased traffic demand in the future.

%In this paper, we have shown that different TE schemes make little or no difference to application performance metrics in today's Internet. This is on account of two reasons: (1) TE schemes mostly optimize link utilization based metrics which are poor predictors of application performance. (2)  Application adaptation due to location diversity in the network increases the capacity of the network for all TE schemes and enables all engineering schemes to achieve near-optimal capacity.


%Based on our results, we question some of the common wisdom related to traffic engineering - TE based on link utlization based metrics; superior performance of online TE compared to offline TE; using MPLS over OSPF for TE. We invite further discussion on these questions.

%We have explored one aspect of TE problem, minimizing congestion in the network. In future work, we plan to study the effect of location diversity on other aspects of TE -  fault tolerance to link failures and ensuring QoS for different classes of traffic in the network.  Current results make us optimistic that location diversity can also benefit these objectives as well.

%We also study the effect of application adaptation on traffic engineering. Especially we study a new type of application adaptation due to  \emph{location diversity}, i.e. content present at multiple locations in the network.  We propose a new metric  to measure the capacity of the network: the relative input output difference of a network. Using ns-2 simulation, we measure the capacity of a network both with and without diversity. We find that location diversity increases the capacity of ISP networks by approx 1.4$\times$ over a network without  location diversity. A surprising consequence of location diversity is that it nearly vanishes the difference among traffic engineering schemes in terms of capacity, e.g., shortest path routing can achieve the same capacity as optimal traffic engineering. Taken together, our findings make a case for simpler traffic engineering schemes in ISP networks .