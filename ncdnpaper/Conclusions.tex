\section{Conclusions}
\label{sec:concl-ncdn}


We posed and studied the NCDN-mangament problem where content delivery and traffic engineering decisions can be optimized jointly by a single entity. %This paradigm is of key importance to \ncp s who own and manage the infrastructure for content distribution as well as the underlying network. 
Our trace-driven experiments using extensive access logs from the world's largest CDN and real ISP topologies resulted in the following key conclusions. First, simple \unplanned\ schemes for routing and placement of NCDN content, such as InvCap and LRU, outperform sophisticated, joint-optimal placement and routing schemes based on recent historic demand. Second,  NCDN traffic demand can be ``shaped'' by effective content placement to the extent that the value of engineering routes for NCDN traffic is small.  Third, we studied the value of the future knowledge of demand for placement and routing decisions. While future knowledge helps, what is perhaps surprising is that a small amount of additional storage allows simple, \unplanned\ schemes to perform as well as \planned\ ones with future knowledge. Finally, with a mix of NCDN and transit traffic, the benefit of traditional traffic engineering is commensurate to the fraction of traffic that is transit traffic, i.e., ISPs dominated by NCDN traffic can simply make do with static routing schemes.  Overall, our findings suggest that content placement is a powerful degree of freedom that NCDNs can leverage to simplify and enhance traditional traffic engineering.



%In this work, we compared different traffic engineering schemes which optimize content placement and/or routing in the network. Our two main findings are as follows; (1) Optimizing content placement alone achieves identical performance to a joint optimization of content placement and routing (2) Optimizing routing while keeping a simple content placement scheme performs sub-optimally. Based on the above two findings we conclude that content placement is more important than routing for optimal trafic engineering.

%In future work, we plan to compare different schemes for the case of varying traffic demands in the network. We also intend to study two additional aspects of our current work: (1) The performance of optimal placement schemes without partial replication which we believe would reduce some benefits of optimizing content placement. (2) Since a real ISP network is likely to have a  larger number of content than in our experiment, we intend to study the case when the number of content in the network is an order of magnitude larger (thousands instead of few hundreds).
