
% We study  \emph{role of content placement strategies on interaction between networks and content delivery systems}, as aspect which has received little attention in prior work in this area. We find that content placement is a strong factor in shaping the flow of  traffic in the network, e.g, placing content at a large number of locations results in localized traffic flows. By choosing the placement strategies suitably, network traffic can be shaped to optimize the objectives of both content delivery systems, e.g., minimizing the latency of files transfers, as well as networks, e.g., minimizing the utilization of links in the network. Further, we show that simple placement schemes are effective in improving cost-, performance- and energy-related metrics for networks and content delivery systems.


%Our goals are to shed light on the role of content placement strategies on interaction between networks and content delivery systems and enhance network architecture support for end-host mobility.

First, and the more pronounced theme, is studying the \emph{role of content placement strategies on interaction between networks and content delivery systems}, as aspect which has received little attention in prior work in this area. We find that content placement is a strong factor in shaping the flow of  traffic in the network, e.g, placing content at a large number of locations results in localized traffic flows. By choosing the placement strategies suitably, network traffic can be shaped to optimize the objectives of both content delivery systems, e.g., minimizing the latency of files transfers, as well as networks, e.g., minimizing the utilization of links in the network. Further, we show that simple placement schemes are effective in improving cost-, performance- and energy-related metrics for networks and content delivery systems.

Second, is \emph{enhancing network architecture support for end-host mobility}.  Today, content delivery systems need to work around the problem of being unable to initiate and maintain connections to mobile hosts by building application-level mechanisms such as push notification services. Our position is that end-host mobility can be seamlessly handled with the support of a global name service that translates host names to their network address under frequent  mobility of end-hosts.  To this end, we present the design, implementation, and evaluation of Auspice, a scalable, geo-distributed name service.  A key decision for this service, affecting both performance and update costs, is the placement of name records of mobile hosts. Auspice infers pockets of high demand for a name and uses a heuristic placement scheme to provide low address lookup latency, low update cost, and high availability. Our evaluation of Auspice's implementation  shows that Auspice significantly outperforms both commercial managed DNS services as well as DHT-based replication alternatives to DNS.


%Our key insight is that interactions between networks and content delivery systems depend strongly on content placement strategies. The locations at which content is placed in the network shapes the flow of  traffic in the network, e.g, placing content at a large number of locations results in localized traffic flows. By choosing the placement strategies suitably, network traffic can be shaped to optimize the desired objectives of content delivery systems, e.g., minimizing the latency of files transfers, as well as networks, e.g., minimizing the utilization of links in the network. 

%This thesis also investigates services a network can offer that are useful in designing content delivery systems.  Much prior research in network architecture has proposed several such services such as QoS, IP multicast, Name-based routing. There is a long list of  possible services and diverse mechanisms are needed to support all of them. Therefore,  we focus on the problem of handling end-host mobility in the Internet, a pressing problem for modern-day Internet in which devices are predominantly mobile, and traffic generated to or from mobile devices is the majority of traffic in the Internet. 


%Therefore, effective content placement not only improves cost and performance metrics of the content delivery systems, but also of the underlying network. 


%This phenomenon implies that networks have an additional knob, in the form of content placement, to 

%Therefor,
%

%Networks have traditionally relied  optimized  routing optimizations 
%Thus content placement provides an additional knob 
%
%While networks have traditionally depended on routing optimizations 
%An implication of this phenomenon is that content placement can be leveraged by the underlying network to 




**************************************************************


 We study the interaction between networks and content delivery in three domains, and find that 
content placement is a strong factor in shaping the flow of  traffic in the network and that simple placement strategies 


 Further, content delivery systems and networks affect each other's decisions in trying to shape the flow of traffic with their respective techniques. Therefore, content delivery systems, networks, and the interaction between the them, all play a role in determining user-perceived performance in the Internet. 



The design of content delivery systems depends on 



As content delivery systems and networks are often under the control of distinct, non-cooperating entities in the Internet, work in one area has largely ignored any interactions with another.

Yet, content delivery systems and networks affect each other's decisions in trying to shape the flow of traffic with their respective techniques. 




A network architecture influences on the design on content delivery systems. 


Networks influence content delivery 

The decisions by networks and content delivery systems 

Both networks and content delivery systems shape the flow of traffic in the network, 

Content delivery systems and networks influence each other's decisions. How?

As content delivery systems and networks are often under the control of distinct, non-cooperating entities in the Internet, work in one area has largely ign


Work in content delivery systems, 



The design and performance of content delivery systems depends on the services that underlying network offers and on network routing decisions.
Reciprocally, content delivery systems affect network routing decisions as they shape the flow of traffic in the network to optimize end-user performance. 
Still, work in content delivery systems and networks has largely ignored the interactions with another, because content delivery systems and networks are often under the control of distinct, non-cooperating entities in the Internet.
The goal of this thesis is to enhance and leverage these interactions to improve cost-, performance-, and energy-related metrics for networks and content delivery systems.


Content delivery systems and networks are often under the control of distinct entities in the Internet. 
As a result, work in content delivery systems and networks has largely ignored the interactions with one another. 
Yet, content delivery systems and networks affect each other's decisions.


 

We consider two approaches for how these interactions occur and how they are leveraged. 

There are two key ideas (1) leveraging the flexibility to place content 

(2) enhancing 

as their respective optimizations 

Content delivery systems and networks both affect each others decisions. 

The services offered by the underlying network 

We consider two forms of interactions between networks and content delivery systems as described below. 

(1) interaction between network route optimization and content delivery decisions such as content placement and request redirections 

(2) how content delivery systems are designed depends on the services offered by the underlying network.




Interaction between networks and content delivery occurs 

This thesis considers two forms of interactions between networks and content delivery. First, 


How content delivery systems are design and 

Content delivery systems rely on connectivity provided by the underlying networks to reach the  

Networks provide the connectivity between hosts across and content delivery systems operate an overlay across 

\begin{enumerate}
\item

*** get feedback from people ***

*** self-revealing bullets ***
\textbf{what is a content delivery network}


*** network CDNs ***
The idea of \emph{content delivery} using a network of distributed caches was pioneered by Akamai in 1990s. Since then, Internet has seen a proliferation in the number of CDNs and their customers and nearly every major web service today relies on a content delivery infrastructure to ensure a high quality experience for its end-users. A typical CDN  operates a set of distributed server deployments, and uses a combination of edge caching, intelligent request redirection, and path and protocol optimizations for delivery of several types of content, e.g., video, bulk downloads, and interactive websites. 

%This thesis takes a broad view of a content delivery network including edge caches and origin servers, and  global and regional CDNs, e.g., a network CDN delivering content to users on its network.

\item
\textbf{content delivery systems need  high performance and cost-effective designs}

Content-delivery networks (CDNs) have always sought to provide a high-quality user experience in a cost-effective manner. Yet, better performing and more cost-effective systems are as relevant for CDNs today as ever, due to following reasons:
(1) Several recent studies have shown that CDNs have potential to attract more traffic by improving the user-perceived performance.
(2) CDN marketplace has seen many new entrants in recent years due to commoditization of CDN technology, which has resulted in increased pressure on CDNs to reduce costs.
(3) CDNs cannot rely on improvements in hardware alone to reduce costs because the size of new content generated and accessed by end-users has outpaced the hardware growth in the last five years, e.g. online video traffic has increased by X hundred\% in last five years and is projected to do so in next 10 years.


%CDNs have more pressure to reduce costs today than  in order to stay competitive in a marketplace that has seen many new entrants in recent years.

%CDNs have increased pressure to reduce costs in order to stay competitive in a marketplace that has seen many new entrants in recent years. Due to these trends,  better performing and more cost-effective systems are as relevant for CDNs today, as ever. 


\item
\textbf{ideas: exploit locality using placement, use planned for predictable workloads}

This thesis uses two key ideas in designing high-performance and cost-effective solutions for CDNs: *** don't introduce terminology *** exploiting workload locality with  suitable placement algorithms (2) for workloads that are predictable over long periods, using a planned approach, i.e., making decisions using analysis of past workloads, to get better results. These ideas have been used in designing CDNs and distributed systems in general. Our contribution is to show application of these ideas in new problem domains and to improve state-of-the-art solutions as described in Section \ref{sec:contribution}. 
We elaborate on these ideas below.


\item
\textbf{content placement is key factor in affecting cost and performance}

User-perceived performance and costs of a CDN depend to a great extent on the content placement algorithms. Placing a content at multiple locations  improves the user-perceived performance but increases infrastructure and operational costs of content delivery system, e.g., placing of large video at multiple locations increases the storage requirements of the system;, placing dynamic objects at multiple locations is associated with costs of propagating updates to all locations; for a CDN that operates its own network, the placement algorithm affects how much traffic is carried by network links hence influences the network costs. 

*** placement, network-effect, ****

*** buzzwords: select buzzwords and use it over and over. ****

CDNs and end-users cooperate 

%Cost-effective and high performance designs can be achieved by leveraging geographical and temporal locality in request patterns by a suitably choosing the placement algorithms.



%Costs incurred by CDNs are mainly due to the costs of replicating objects at multiple locations. 

\item
\textbf{content placement exploits locality to improve cost and performance}

Intelligent content placement strategies would be of limited use in designing high performance, cost-effective solutions for CDNs if  there was no geographic and temporal locality to requests.  Real workloads do show geographic and temporal locality patterns as is shown in experiments in this thesis as well as in several measurement studies of web, video, and social networking traffic. Intelligent placement strategies can exploit the locality in workload by placing objects, at those times and locations, where the content is highly popular. Thus, an effective placement policy can enable a CDN to limit the cost of replicating objects, and improve user-perceived performance. 



\item
\textbf{Design space: planned and unplanned schemes}

We consider two broad classes of solutions in this thesis: \emph{planned}  and \emph{unplanned}.
%This thesis studies two approaches to resource management: \emph{planned} and \emph{unplanned}. 
A planned approach assumes that  workloads in future will resemble those in the past, and makes decisions including those of  placement, redirection,  routing, and bandwidth allocation using an historical knowledge of workloads. An unplanned approach makes decisions either using a static policy or using an online algorithm based on local information. 
While a planned approach can potentially use its long-term knowledge of workloads to outperform unplanned approaches, it is associated with the overhead of collecting workload information and computing solutions using a global knowledge of workloads. 
In this thesis, we study planned and unplanned approaches for a variety of problems, to understand the performance and overhead of both types of solutions. 
A planned approach is not  effective for all problems we consider, either because of poor workload predictability, or due to overhead of executing a planned approach. In some scenarios, we find that  a planned approach can indeed be executed with a small overhead, and gives significant benefits over an unplanned approach. 

%In this thesis, we find scenarios where a planned approach is unusable either because of poor workload predictability, or due to overhead of executing a planned approach. For other problems, 

%In this thesis, we study planned and unplanned approaches for a variety of problems, and find that in some cases, a planned approach can indeed be executed with a small overhead, and gives significant benefits over an unplanned approach.

%For predictable workloads, if a planned approach can be executed with a small overhead, we show that a planned approach gives significant benefits over an unplanned approach. 


%Table 1 gives examples of planned and unplanned solutions. 

%We guide our design based on answers to the following questions that are broadly applicable to all problems:  (1) Whether a planned approach incurs a small overhead and is simple to implement? (2) Does a planned approach outperform an unplanned approach? (3) How well does a \emph{semi-planned approach} compare to a purely planned on or a purely unplanned strategy? (4) How does a planned or an unplanned approach compare to a \emph{perfectly planned approach} with an exact knowledge of workload? 



\item
\textbf{thesis goal}

This goal of this thesis is to present research on  **propose not the right word ** cost-effective and high performance content delivery systems  and 
conduct experimental and data-driven evaluation 

of the proposed and several existing systems in the context of (1) an Internet service provider network with location diversity of content  (2) a network CDN, i.e., an Internet service provider running a CDN on its network (3) a CDN replicating highly dynamic data objects, (4) a CDN using peer-to-peer technology to minimize bandwidth costs, and  (5) a CDN seeking to minimize infrastructure energy use.

*** our contributions:   ***

*** make references ***


We develop a diverse set of solutions to (1) optimize different types of  cost metrics, e.g. bandwidth costs, energy costs, and network infrastructure costs, (2) design workload-specific solutions, e.g., caching algorithms used by static content are less useful for highly dynamic content, and (3) address differences in CDN technology, e.g. peer-to-peer CDNs require a different bandwidth allocation policy from infrastructure-based CDNs. *** different ***



\section{Research questions and contributions}
\label{sec:contribution}



*** carefully select opinions ***

start with random 

placement of dynamic objects



proposed work.

 

*** put abstract ***


*** distinguished proposed vs done ***

*** where published ***


\textbf{ISP-CDN interaction:} *** placement is random, stress that ***

**** use content delivery instead of CDN ****


How does routing optimization by ISPs to minimize network costs affect application performance considering the effect of location diversity of content, and also how does application adaptation to location diversity of content affect routing optimizations done by ISPs?


\textbf{Network CDN management:}
What is best strategy for a network CDN, a CDN operated by an ISP on its network, to optimize content placement, routing, and redirection, in order to reduce network costs and user-perceived performance?

\textbf{Dynamic data replication:}
How do we design a system for placing dynamic objects across the globe in order to minimize lookup latencies to those objects and reduce the costs of updating those objects?

\textbf{Bandwidth-cost minimization:}
If a CDN were to use P2P technology, how should it allocate bandwidth among multiple content to optimize performance and cost objectives?


\textbf{Energy minimization:}
How can a CDN optimize placement, routing and redirection within in an edge cluster to minimize energy use of the cluster with minimal effect on user-perceived performance? 

*** self-



%Further, a planned approach can improve its decisions over time, by learning from decisions done in the past. 

%A planned approach can possibly outperform an unplanned approach due to a global optimization of resources. But, its effectiveness is subject to  the predictability of the workload, and whether it is feasible to estimate the globally optimal solution or even an approximation to the optimal solution. 

%The advantage of an unplanned approach over a planned approach is its simplicity, as it does not depend on a long term knowledge of content workloads. 


%Potentially, a planned approach can outperform an unplanned approach due to a global optimization of resources and a long-term knowledge of workloads. However, we find that simple unplanned approaches perform well in many cases. In some cases, a planned approach indeed gives better performance than an unplanned approach, but a planned approach is not always effective due to unpredictability of the workload or because it is infeasible to estimate the optimal solution or even its approximation.





\end{enumerate}
