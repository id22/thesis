%!TEX root = New.tex
\begin{abstract}

%The Internet's tremendous success as well as our maturing realization of its architectural shortcomings have attracted significant research attention towards clean-slate re-designs in recent times. A number of these shortcomings can be traced back to {\em naming} issues, especially in the manner IP addresses conflate identity and network location, e.g., the poor support for mobility and multihoming; vulnerability to hijacking and spoofing of addresses; the inability of a receiver to control traffic; etc.

Mobile devices dominate the Internet today, however the Internet continues to operate in a manner similar to its early days with poor infrastructure support for mobility. Our position is that in order to address this problem, 
a key challenge that must be addressed is the design of a massively scalable global name resolution infrastructure that rapidly resolves identities to network locations under high mobility. Our primary contribution is the design, implementation, and evaluation of \auspice, a next-generation global name resolution service that addresses this challenge. A key insight underlying \auspice\ is a {placement engine} that replicates name records to provide low lookup latency,  low update cost, and high availability. %\auspice\ employs a heuristic placement algorithm that determines the number and locations of resolver replicas so as to reduce user-perceived response times and update cost while ensuring that no replica location becomes a load hotspot. 
We have implemented a prototype of \auspice\ and compared it against several commercial managed DNS services as well as alternative replication strategies. Further, we present a proof-of-concept validation of end-to-end mobility and novel case-study apps leveraging \auspice's support for context-aware communication that generalize address- or name-based communication. 

%We have implemented a prototype of \auspice\ and evaluated it on Planetlab, a local cluster, as well as through large-scale trace-driven experiments. Our experiments show that \auspice\ provides 1.0 sec to 24.7 sec lower update latencies than commercial managed DNS services and up to 9$\times$ lower lookup latencies than a popularity-based DHT replication scheme. 

%well as  to DNS by up to an order of magnitude.

%which places replicas of name resolvers close to pockets of demand so as to optimize user-perceived response times while respecting capacity and consistency constraints. Furthermore, we also develop simple heuristic algorithms that achieve response times close to the optimal. To demonstrate the generality of \auspice's service replica placement engine, we conduct several case studies including a name resolution workload based on traces from Alexa and a large large commercial CDN; a Twitter stream service; and a compute-intensive database querying service. Our experiments based on an implemented prototype of \auspice\ over Planetlab and Amazon EC2 as well as larger-scale trace-driven experiments show that \auspice\ achieves significant improvements in user-perceived response times, in some cases exceeding an order of magnitude compared to state-of-the-art solutions.

\end{abstract}
